\section{Technologie Auswahl}

Dieses Kapitel behandelt die Auswahl der Frontend-Technologie für die Umsetzung der Formular-Anwendung. Dazu  werden im ersten Schritt die dafür in Frage kommende Technologien identifiziert.  Anschließend wird der Trend der Popularität dieser Technologien miteinander verglichen. Die daraus resultierenden Kandidaten sollen dann einem detaillierter untersucht werden. In Hinblick auf die Anforderungen an die Formular-Anwendung soll dabei die angemessenste Frontend-Technologie ausgewählt werden.

\subsection{Trendanalyse}

Zwei Quellen wurden für die Analyse der Technologie-Trends ausgewählt: die Ergebnisse der jährlichen Stack Overflow Umfragen und das Such-Interesse von Google Trends. 

\paragraph{Stack Overflow Umfrage}
Die Internet-Plattform Stack Overflow richtet sich an Softwareentwickler und bietet ihren Nutzern die Möglichkeiten, Fragen zu stellen, Antworten einzustellen und Antworten anderer Nutzer auf- und abzuwerten. 
Besonders für Fehlermeldungen, die häufig während der Softwareentwicklung auftreten, findet man auf dieser Plattform rasch die Erklärung und den Lösungsvorschlag gleich mit. Dadurch lässt sich auch die Herkunft des Domain-Namens herleiten:

\begin{quotation}
We named it Stack Overflow, after a common type of bug that causes software to crash -- plus, the domain name stackoverflow.com happened to be available. - Joel Spolsky, Mitgründer von Stack Overflow \footnote{\cite{TheUnprovenPath}}
\end{quotation}

Aufgrund des Erfolgsrezepts von Stack Overflow ist die Plattform kaum einem Softwareentwickler unbekannt. Dementsprechend nehmen auch jährlich tausende Entwickler an den von Stack Overflow herausgegebenen Umfragen teil. Seit  2013 beinhaltet die Umfragen auch die Angabe der aktuell genutzten und in Zukunft gewünschten Frontend-Technologien.
Stackoverflow erstellt aus diesen gesammelten Daten Auswertungen und Übersichten. Doch gleichzeitig werden die zugrundeliegenden Daten veröffentlicht. \footnote{\cite{StackOverflowInsights}} 

Um den Trend der Beliebtheit der Frontend-Technologien aufzuzeigen, wurde ein Jupyter Notebook erstellt. Es transformiert die Daten in ein einheitliches Format, da die  Umfrageergebnisse von Jahr zu Jahr in einer unterschiedlichen Struktur abgelegt wurden. Anschließend erstellt es Diagramme, die im Folgenden analysiert werden. Das Jupyter Notebook ist im  Anhang zu finden.

\paragraph{Google Trends} Suchanfragen die an die Suchmaschine Google  abgesetzt werden, lassen sich  über den Dienst Google Trends  als Trenddiagramm Visualisieren. Um das relative Such-Interesse abzubilden werden die Ergebnisse normalisiert um die Ergebnisse auf einer Skala von 0 bis 100 Darstellen zu können. \footnote{Vgl. \cite{GoogleTrendsHilfe}}

\begin{quotation}
Google Trends ist keine wissenschaftliche Umfrage und sollte nicht mit Umfragedaten verwechselt werden. Es spiegelt lediglich das Suchinteresse an bestimmten Themen wider. \footnote{\cite{GoogleTrendsHilfe}}
\end{quotation}

Genau aus diesem Grund wird Google Trends im folgenden lediglich zum Abgleich der Ergebnisse der Stackoverflow Umfrage  eingesetzt.


\subsubsection{Eingestellte Projekte}



\subsubsection{Stack Overflow Umfrage}
