
\subsection{Vergleich React Native und Flutter}

\subsubsection{Automatisiertes Testen}

\paragraph{Automatisierte Test in React Native} Die React Native Dokumentation führt genau eine Seite mit einem Überblick über die unterschiedlichen Testarten. Dabei wird das Konzept von Unit Tests, Mocking, Integrations Tests, Komponenten Tests und Snapshot Tests kurz erläutert, jedoch ohne ein Beispiel zu geben oder zu verlinken. Vier Quellcodeschnipsel sind auf der Seite zu finden: Ein Schnipsel zeigt den minimalen Aufbau eines Tests, zwei weitere Schnipsel veranschaulichen Beispielhaft, wie Nutzerinteraktionen getestet werden können und Letzteres zeigt die textuelle Representation der Ausgabe einer Komponente, die für einen Snapshottest verwendet wird. Weiterhin wird auf die Jest API Dokumentation verwiesen, sowie auf ein Beispiel für einen Snapshot Test in der Jest Dokumentation. [https://reactnative.dev/docs/testing-overview]

Um die notwendigen Anleitungen für das Erstellen der jeweiligen Tests ausfindig zu machen, ist es notwendig, die Dokumentation von React Native zu verlassen.

Die Dokumentation von Jest enthält mehr Details zum Einsatz der Testbibliothek, welches für mehrere auf Javascript basierende Frontend Framework kompatibel ist. [https://jestjs.io/docs/getting-started]. Somit muss zum Erstellen der Unit-Tests immerhin nur dieses Framework studiert werden.

Zum Entwickeln von Test von React Native Komponenten  wird unter anderem auf die Bibliothek React Native Testing Library verwiesen. Anders als der Name vermuten lässt, handelt es sich nicht um eine von React Native bereitgestellte Bibliothek. Stattdessen wird sie vom Drittanbieter Callstack angeboten. [https://callstack.github.io/react-native-testing-library/docs/getting-started/]
Sie verwendet im Hintergrund den React Test Renderer, welcher wiederum vom React Team angeboten wird und auch zum Testen von react.js Anwendungen geeignet ist. [https://reactjs.org/docs/test-renderer.html]. Der React Test Renderer Wird ebenfalls empfohlen, um Komponententest zu kreieren, die keine React Native spezifischen Funktionalitäten nutzen.


Um Integrationstest  so entwickeln, welche die Applikation auf einem physischen Gerät oder auf einem Emulator testen, wird auf zwei weitere Drittanbieter Bibliotheken verlinkt: Appium [http://appium.io/] und Detox [https://github.com/wix/detox/]. Es wird darauf hingewiesen, dass Detox speziell für die Entwicklung von React Native Integrationstest entwickelt wurde. Appium wird lediglich als ein weiteres  bekanntes Werkzeug erwähnt. 

Es lässt sich damit zusammenfassen, dass der Aufwand der Einarbeitung für automatisiertes Testen in React Native vergleichsweise hoch ist. Die Dokumentation ist auf die Seiten der jeweiligen Anbieter verteilt. Der Entwickler muss sich den Überblick selbst verschaffen und zusätzlich die für das Framework React Native relevanten Inhalte identifizieren. Notwendig ist auch das Erlernen von mehreren APIs um alle Testarten abzudecken. Für einen Anfänger kommt erschwerend hinzu, dass eine Entscheidung für die eine oder andere Bibliothek notwendig wird. Um diese Entscheidung treffen zu können, ist eine Auseinandersetzung mit den Vor- und Nachteile der Technologien im Vorfeld vom Entwickler zu leisten.




\paragraph{Automatisierte Test in Flutter} Die Flutter Dokumentation erklärt sehr umfangreich auf 11 Unterseiten die unterschiedlichen Testarten mit Quellcodebeispielen und verlinkt für jede Testart eine bis mehrere detaillierte Schritt-für-Schritt-Anleitungen, wie ein solcher Test erstellt wird.

Einer Seite erklärt den Unterschied zwischen Unit Test, Widget Test und Integrationstest [https://flutter.dev/docs/testing]. Eine weitere Seite erklärt Integrationstests in mehr Detail [https://flutter.dev/docs/testing/integration-tests]. 

Ein sogenanntes Codelab führt durch die Erstellung einer minimalistischen App und zwei Unit-, fünf Widget- und zwei Integrationstest für diese App [https://codelabs.developers.google.com/codelabs/flutter-app-testing]

Im sogenannten Kochbuch tauchen folgende Rezepte auf:

\begin{itemize}
    \item 2 Rezepte für Unit Tests
    \begin{itemize} 
       \item eine grundlegende Anleitung zum Erstellen von Unit-Tests [https://flutter.dev/docs/cookbook/testing/unit/introduction]
       \item Eine weitere Anleitung zum Nutzen von mocks in Unit Test mithilfe der Bibliothek mockito [https://flutter.dev/docs/cookbook/testing/unit/mocking]
    \end{itemize}
    \item 3 Rezepte für Widget Tests
    \begin{itemize} 
        \item Eine grundlegende Anleitung zum Erstellen von Widget Test [https://flutter.dev/docs/cookbook/testing/widget/introduction]
        \item Ein Rezept mit detaillierteren Beispielen zum Finden von widgets  zur Laufzeit eines Widget Tests [https://flutter.dev/docs/cookbook/testing/widget/finders]
        \item Ein Rezept zum Testen von Nutzerverhalten wie dem Tab, dem Drag und dem eingeben von Text [https://flutter.dev/docs/cookbook/testing/widget/tap-drag]
     \end{itemize}
    \item 3 Rezepte für Integrationstests
    \begin{itemize} 
        \item Eine grundlegende Anleitung zum Erstellen eines Integrationstest [https://flutter.dev/docs/cookbook/testing/integration/introduction]
        \item eine Anleitung zum simulieren von scrollen in der Anwendung während der Laufzeit eines Integrationstest [https://flutter.dev/docs/cookbook/testing/integration/scrolling]
        \item eine Anleitung zum Performance Profiling [https://flutter.dev/docs/cookbook/testing/integration/profiling]
     \end{itemize}
\end{itemize}

Zusammengefasst: Der Aufwand der Einarbeitung in das Testen in Flutter ist gering. Alle Werkzeuge werden vom Dart- und Flutter-Team bereitgestellt. Die Dokumentation ist umfangreich, folgt jedoch einem roten Faden. Eine Übersichtsseite fasst die Kerninformationen zusammen und verweist auf die jeweiligen  Seiten für detailliertere Informationen und Übungen.