
\newpage
\thispagestyle{empty}

Hochschule Harz\newline
Fachbereich Automatisierung und Informatik
\vfill
\begin{center}

\large{\textsc{Thema und Aufgabenstellung der Masterarbeit}}


\large{\textsc{MA AI ??/2021}}

\vfill

\large{\textsc{für Herrn Alexander Johr}}

\vfill

%\vfill

\large{\textsc{Entwicklung einer Formularanwendung mit Kompatibilitätsvalidierung der Einfach- und Mehrfachauswahl-Eingabefelder}}



\end{center}

\vfill

%gut darlegen könnendie methode spielt mehr die rolle


%andeuten nicht ins detail

%ich möchte mich testing


%kern des themas

%immer zurück kommen

%höhere eigen zum ergebis

%das liebesleben der tennisbällen unter einfluss des mondscheins

%größtmögliche unfall der passiert mal wahrscheinlichkeit 

%kompliziertheit komplexität

%werner von braun
%triebwerke geschrottet

%doku 
%arte 3 sat
%allgemein menschheit in den weltraum

%feindifferenziert


%Frage an Herr Ackermann / Singer: Vergleich mit Angular Dart, Angular allgemein, XAML WPF, evtl. etwas 

%Aber nicht alle

%Deshalb Fallstudie, richtig?



Das Thünen-Institut für Ländliche Räume erhebt Daten von Maßnahmen
 des Europäischen
Landwirtschaftsfonds für die Entwicklung des ländlichen Raums
 (ELER).
Um die Eingabe für die Wissenschaftler des Instituts zu beschleunigen
 und um fehlerhafte Eingaben zu minimieren, soll eine 
 spezielle Formularanwendung entwickelt werden.
Neben herkömmlichen Freitextfeldern beinhaltet das gewünschte Formular zum Großteil Eingabefelder für Einfach- und Mehrfachauswahl.
Je nach Feld kann die Anzahl der Auswahloptionen mitunter zahlreich sein.
Dem Nutzer sollen daher nur solche Auswahloptionen angeboten werden,
die zusammen mit der zuvor getroffenen Auswahl sinnvoll sind.

\vspace{14pt}

Im Wesentlichen ergibt sich die Kompatibilität 
der Auswahloptionen aus der Bedingung, 
dass für dasselbe oder ein anderes Eingabefeld eine Auswahlmöglichkeit gewählt bzw.
nicht gewählt ist. Diese Bedingungen müssen durch 
Konjunktion und Disjunktion verknüpft werden können.
In Sonderfällen muss ein Formularfeld jedoch auch 
die Konfiguration einer vom Standard abweichenden Bedingung
ermöglichen. 
Wird dennoch versucht,
eine deaktivierte Option zu selektieren, wäre eine Anzeige der
inkompatiblen sowie der stattdessen notwendigen Auswahl ideal.

\vspace{14pt}
Die primäre Zielplattform der Anwendung ist das Desktop-Betriebssystem
Microsoft Windows 10.
Idealerweise ist die Formularanwendung auch auf weiteren Desktop-Plattformen sowie
mobilen Endgeräten wie Android- und iOS-Smartphones und -Tablets
lauffähig. Die Serialisierung der eingegebenen Daten genügt dem Institut 
zunächst als Ablage einer lokalen Datei im JSON-Format. 
%Nach der Testphase des Eingabeformulars erfolgt der Import der Daten in das Relationale Datenbank-Management System ProstgreSQL. 

% Beim Versuch, eine inkompatible Auswahl zu selektieren,
% könnte die Visualisierung eines Mengendiagramms der erforderlichen und invaliden
% Auswahlfelder und die User Experience weiter verbessern.
% Als Ergebnis der Anforderungsanalyse wurde 
% ein UI-Framework ausgewählt,
% welches neben weiteren Auswahlkriterien 
% insbesondere ein hohes Maß an Flexibilität erlaubt: Flutter.





%Die freiheiten in der Oberflächenentwicklung 


%Ziel dieser Masterarbeit ist die Entwicklung der Formular-Anwendung 
%und dabei den Entwicklungsschitteprozess und die 

%Flutter bietet arbeitet mit funktionaler UI. Damit gibt Flutter die Wahl des Zustandsmanagements.


%Dies bietet einige Vorlteile. Koontrolle über die Performance.
%Vereinfachte Fehlersuche in der UI. Freiheiten in der Entwicklung der UI.
%Konfiguration von wiederverwendbaren Oberflächenkomponenten mithilfe des Strategy Entwurfsmusters.
%Doch es ergeben sich auch Nachteile: 
%Es gibt keine goto Methode für zustandsmanagement sondern eine Auswahl der 
%richtigen Methoe muss auf Grundlage der Anfforderungen gewählt werden.

%Zier der Arbeit ist es die Vorteile und Nachteile näher zubeleuchten.
%Dies soll mit der Entwicklung einer Anwendung in Flutter begleitet werden. Die Anwendung soll eine komplexes Eingabe-Formular für das Thünen institut in Braunschweig sein.



%Templates <-> Functional UI

%Exceptions <-> Hot Reload

%Stragety Pattern

%Reguläre ausdrücke

%Refactoring

%Programmgenerierung JSON AOT vs C\# Reflection


%Kontrolle

%Nachteile

%Provider BloC <-> Provider
%Eine  Data Binding

%Die zwei b Die beiden 

%Performance

%Die Produkte QlikView und Qlik Sense - Business Intelligence Software des Software\-unter\-nehmens Qlik - bieten ihren Anwendern mit einer Reihe an unterschiedlichen Diagramm\-typen einen Überblick über ihre Geschäftsdaten mittels Ad-hoc-Analysen. Nicht alle Wünsche der Anwender lassen sich über die Konfigurations\-möglich\-keiten dieser vorgefertigten Diagramme abdecken. Eine Alternative stellen die sogenannten Extension Objects und Document Extensions dar, die mit mithilfe von Webtechnologien wie JavaScript, HTML, CSS und XML entwickelt werden können. Die Entwicklung von Programmen mit JavaScript erweist sich jedoch gegenüber der Entwicklung mit anderen Programmiersprachen als sehr mühsam.


%Ziel dieser Bachelorarbeit ist es die Entwicklung dieser Extension Objects sowie Document Extensions mit der Programmiersprache Dart von Google umzusetzen, da sich diese für die Entwicklung skalierbarer und strukturierter Webanwendungen eignet. Die Arbeit bietet einen Leitfaden wie solche Extensions mit Dart entwickelt werden können und welche Vor- und Nachteile dies gegenüber der Entwicklung mit JavaScript bietet. 


%Die Bachelorarbeit beinhaltet folgende Teilaufgaben:
%\begin{itemize}
%	\itemsep0em
%	\item Analyse der Unterschiede von QlikView 11 und Qlik Sense bei der Entwicklung von Extension Objects sowie von Document Extensions
%	\item Analyse der Einschränkungen von Extensions Objects gegenüber der von QlikView 11 und Qlik Sense mitgelieferten Sheet Objects
%	\item Analyse der Auswirkungen der Extensions auf die Performance
%	\item Analyse der Vor- und Nachteile der Entwicklung von Extensions mit Dart im Vergleich zur Entwicklung von Extensions mit JavaScript
%	\item Entwicklung von zeitsparenden Methoden zur Entwicklung von Extensions
%	\item Bewertung der Ergebnisse
%\end{itemize}

\vspace{14pt}
Die Masterarbeit umfasst folgende Teilaufgaben:
\begin{itemize}
    \itemsep0em
\item Analyse der Anforderungen an die Formularanwendung
\item Evaluation der angemessenen Technologie für die Implementierung
\item Entwurf und Umsetzung der Übersichts- und Eingabeoberfläche
\item Konzeption und Implementierung der Validierung der Eingabefelder
\item Entwicklung von automatisierten Testfällen zur Qualitätskontrolle
\item Bewertung der Implementierung und Vergleich mit den Wunschkriterien
\end{itemize}


%Die Freiheiten der Oberflächenentwicklung in Flutter sollen genutzt werden, um 
%visuellen Komponenten zu erstellen, die durch Einsatz des Strategie-Entwurfsmusters
%im hohen Maße anpassbar sind.



%Da die Aktualisierung von Flutter-Views nicht automatisch erfolgt, muss ein angemessenes 
%Zustandsmanagement evaluiert und eingesetzt werden.
%Eine angemessene Technologie zur Serialisierung soll gewählt und 
%das Persistieren der eingegebenen Datensätze damit umgesetzt werden.
%Durch die Menge der Auswahlfelder ist bei Weiterentwicklung der APIs mit 
%hohem Migrationsaufwand der Codebasis zu rechnen. Der Einsatz von regulären Ausdrücken
%soll helfen, den Prozess zu automatisieren. Die Entwicklung automatisierter Testfälle
%soll weiterhin ermöglichen, Fehler bei der Weiterentwicklung zeitnah zu erkennen.



%Die Masterarbeit beinhaltet damit folgende Teilaufgaben:
%\begin{itemize}
%	\itemsep0em
%	\item Evaluation des Zustandsmanagements und Implementierung des View Models
%    \item Entwicklung von wiederverwendbarer Komponenten des Views und Anpassung dieser mit dem Strategie Entwurfsmuster
%    \item Auswahl der Technologie zur Serialisierung und Implementierung der Persistierung des Models
%	\item Migration der Codebasis auf aktualisierte APIs mittels Regulärer Ausdrücke
%	\item Entwicklung von Automatisierten Testfällen
%\end{itemize}

\vfill
\vfill

\begin{tabularx}{\textwidth}{@{} *2{>{\centering\arraybackslash}X}@{}}
Prof. Jürgen Singer Ph.D. & Prof. Daniel Ackermann \\
1. Prüfer                 & 2. Prüfer	 \\
\end{tabularx}	     