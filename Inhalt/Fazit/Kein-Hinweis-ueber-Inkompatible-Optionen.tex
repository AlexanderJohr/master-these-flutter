\chapter{Anzeige von fehlerhaften Teilkomponenten der Bedingungen von deaktivierten Auswahloptionen}
\label{chap:Anzeige-von-fehlerhaften-Teilkomponenten-der-Bedingungen-von-deaktivierten-Auswahloptionen}


Einen Wunschkriterium für die Formularapplikation war es,
bei der Auswahl von deaktivierten Optionen einen Hinweise zu erhalten,
warum diese deaktiviert ist.

In Kapitel \ref{Schritt4} ist die Umsetzung der Deaktivierung von Optionen beschrieben.
Eine Funktion zur Überprüfung der Bedingung einer Optionen wird der Option bei dessen Erstellung im Konstruktor übergeben.
Sie wird bei Überprüfung der Kompatibilität der Auswahloption mit den restlichen im Formular ausgewählten Optionen ausgeführt.
Die Konjunktion, Disjunktion und Negation wird mit den Operatoren für das logische Und und das logische Oder sowie das logische Nicht umgesetzt.
Doch auf diese Art und Weise ist es nicht möglich,
herauszufinden,
welche der einzelnen Abfragen zu einem Fehler führte.
Auf den Inhalt der Funktion kann zur Laufzeit nicht zugegriffen werden.
Die Einzelkomponenten der Bedingung sind damit also nicht bekannt. 
Es ist daher nur möglich,
auf die Komponenten der Bedingung zuzugreifen,
wenn die gesamte Bedingung als eine Datenstruktur abgelegt ist.
Diese Datenstruktur muss die Konjunktion, Disjunktion und Negation unterstützen.

Die Konzeption und Implementierung einer solchen Datenstruktur und des dazugehörige Algorithmus zur Identifizierung der inkompatiblen Komponenten bedarf einer intensiven wissenschaftlichen Recherche und Ausarbeitung.
Als Wunschkriterien steht diese Funktion somit nicht im Kosten-Nutzen-Verhältnis, weshalb sich gegen die Ausarbeitung in dieser wissenschaftlichen Arbeit entschieden wurde.

