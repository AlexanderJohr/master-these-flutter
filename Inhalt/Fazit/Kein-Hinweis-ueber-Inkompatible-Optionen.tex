\section{Anzeige von fehlerhaften Teilkomponenten der Bedingungen von deaktivierten Auswahloptionen}
\label{sec:Anzeige-von-fehlerhaften-Teilkomponenten-der-Bedingungen-von-deaktivierten-Auswahloptionen}


Ein Wunschkriterium für die Formularapplikation war es,
bei der Auswahl von einer deaktivierten Option einen Hinweis zu erhalten,
warum diese deaktiviert ist.

In Kapitel \ref{Schritt4} ist die Umsetzung der Deaktivierung von Optionen beschrieben.
Die Option validiert sich selbst und bekommt zu diesem Zweck eine Funktion übergeben. Diese Funktion überprüft die Kompatibilität mit allen anderen Feldern im Formular.
Konjunktion, Disjunktion und Negation werden mit den Operatoren für das logische Und und das logische Oder sowie das logische Nicht umgesetzt.
Doch auf diese Art und Weise ist es nicht möglich,
herauszufinden,
welche der einzelnen Abfragen zu einem Fehler führte.
Auf den Inhalt der Funktion kann zur Laufzeit nicht zugegriffen werden.
Die Einzelkomponenten der Bedingung sind damit also nicht bekannt. 
Es ist daher nur möglich,
auf die Komponenten der Bedingung zuzugreifen,
wenn die gesamte Bedingung als eine Datenstruktur abgelegt ist.
Diese Datenstruktur muss die Konjunktion, Disjunktion und Negation unterstützen.
Eine Lösung könnte die Nutzung eines \enquote{Und-Oder-Baums} zusammen mit einem sogenannten \enquote{Branch-and-Bound}-Algorithmus -- deutsch Verzweigung und Schranke -- sein.
Diese Vorgehensweise könnte erlauben, die Bedingung einer Option in mehrere Teilbedingungen zu verzweigen und durch den dadurch aufgespannten Baum zu traversieren. \HP{REF}
% https://en.wikipedia.org/wiki/Branch_and_bound
%https://www.jstor.org/stable/1910129?origin=crossref
% An Automatic Method of Solving Discrete Programming Problems


Die Konzeption und Implementierung einer solchen Datenstruktur und des dazugehörigen Algorithmus zur Identifizierung der inkompatiblen Komponenten bedürfen einer intensiven wissenschaftlichen Recherche und Ausarbeitung.
Als Wunschkriterium steht diese Funktion somit nicht im Kosten-Nutzen-Verhältnis, weshalb sich gegen die Ausarbeitung in dieser wissenschaftlichen Arbeit entschieden wurde.

