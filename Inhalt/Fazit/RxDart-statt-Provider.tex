 
\section{Reevaluation des Zustandsmanagements}
\label{sec:Reevaluation-des-Zustandsmanagements}


Während der Implementierung wurde eine passende Vorgehensweise gesucht,
um den Zustand der Applikation zu verwalten
und damit die Aktualisierung der Oberfläche auszulösen.
Für simple Applikationen empfiehlt Google den integrierten Mechanismus der \enquote{StatefulWidgets} und deren Methode \enquote{setState} zu verwenden \footcite[Vgl.][]{AddingInteractivityToYourFlutterApp}.
Doch durch die hohe Anzahl der Oberflächenelemente in der finalen Applikation ist diese Vorgehensweise nicht empfehlenswert.
Sie setzt das Aktualisieren gesamter Widgets bei Anpassung des Zustandes voraus,
was für die Laufzeitgeschwindigkeit die intensivste Belastung darstellt.
Stattdessen wurde versucht,
einem Mechanismus zu verwenden,
der es erlaubt,
nur Teile der Oberfläche neuzuzeichen,
die wirklich eine Aktualisierung benötigen.

Zu diesem Zweck empfiehlt Google das Nutzen des Pakets \enquote{provider} der Flutter Community \footcite[Vgl.][]{ProviderARecommendedApproach}.
Dieser Ansatz wurde in der Implementierung ursprünglich verwendet.
Das Paket hat den Nachteil,
dass für jeden Zustand,
der die Aktualisierung eines Teils der Oberfläche bewirken soll,
eine neue Klasse erstellt werden muss, die von \IC{ChangeNotifier} erbt.
Eine Möglichkeit ist,
dass jede dieser Klassen den nötigen Boilerplate-Quellcode enthält,
welcher die Oberfläche über die Methode \IC{notifyListeners} benachrichtigt.
Eine andere Möglichkeit ist es, für den gleichen Datentyp den benötigten BoilerplateCode in einer eigenen Basisklasse auszulagern
und dann von dieser Klasse zu erben wie in Listing \label{lst:ChoiceChangeNotifier} zu sehen.
\ifIncludeFigures
  \begin{listing}[h]
    \begin{minted}[]{dart}
class ChoiceChangeNotifier extends ChangeNotifier {
    BuiltSet<Choice> _choices = BuiltSet<Choice>();

    BuiltSet<Choice> get choices => _choices;

    set choices(BuiltSet<Choice> choices) {
        _choices = choices;
        notifyListeners();
    }
}
class LetzterStatusViewModel extends ChoiceChangeNotifier {}
\end{minted}
    \caption[built_value Live Template]{Live Template für die Erstellung von built_value Boilerplate-Code in Android Studio, Quelle: Jetbrains Marketplace Built Value Snippets Plugin}
    \label{lst:ChoiceChangeNotifier}
  \end{listing}
\fi
\IC{ChoiceChangeNotifier} verwaltet den internen privaten Zustand \IC{_choices} \Z{3} über die öffentlichen Schnittstellen zum Lesen \Z{4} und Schreiben \Z{6-9}.
Bei Aktualisierung des Wertes erhalten alle Listener eine Benachrichtigung \Z{8}.
\IC{LetzterStatusViewModel} erbt dieses Verhalten, doch hat die Klasse darüber hinaus keine Implementierung.

Anschließend muss jeder \IC{ChangeNotifier} als ein \IC{ChangeNotifierProvider} registriert werden \LstZ{\ref{lst:MultiProvider}}{7}. 
Der \IC{MultiProvider} kann genutzt werden, um mehrere Provider in einer Liste zu übergeben.
Dort werden auch andere Services wie etwa \IC{MassnahmenFormViewModel} \Z{3} und \IC{MassnahmenModel} \Z{6} hinterlegt.

\ifIncludeFigures
  \begin{listing}[h]
    \begin{minted}[]{dart}
MultiProvider(
  providers: [
    Provider<MassnahmenFormViewModel>(create: (_) => MassnahmenFormViewModel()),
    Provider<MassnahmenJsonFile>(create: (_) => MassnahmenJsonFile()),
    Provider(
        create: (context) => MassnahmenModel(
            Provider.of<MassnahmenJsonFile>(context, listen: false))),
     ChangeNotifierProvider(create: (context) => LetzterStatusViewModel())),
  ],
  child: MaterialApp(),
)
\end{minted}
    \caption[built_value Live Template]{Live Template für die Erstellung von built_value Boilerplate-Code in Android Studio, Quelle: Jetbrains Marketplace Built Value Snippets Plugin}
    \label{lst:MultiProvider}
  \end{listing}
\fi

Dann ist der \IC{ChangeNotifier} in dem Widget, welches den Parameter \IC{child} übergeben wird und darüber hinaus allen Kindern-Elementen dieses Widgets verfügbar.
Über einen \IC{Consumer} kann in der Oberfläche auf Änderungen des \IC{ChangeNotifier} reagiert werden \Lst{\ref{lst:Consumer}}.

\ifIncludeFigures
  \begin{listing}[h]
    \begin{minted}[]{dart}
  Consumer<LetzterStatusViewModel>(
      builder: (context, choiceChangeNotifier, child) {
      },
    )
\end{minted}
    \caption[built_value Live Template]{Live Template für die Erstellung von built_value Boilerplate-Code in Android Studio, Quelle: Jetbrains Marketplace Built Value Snippets Plugin}
    \label{lst:Consumer}
  \end{listing}
\fi


Doch diese Vorgehensweise bietet im Vergleich zu den von Flutter mitgelieferten \enquote{Widgets} keine Vorteile.
Das Äquivalent zum \IC{Consumer} ist das mitgelieferten Widget \IC{StreamBuilder},
welcher mit jeder Art von \enquote{Stream} verwendet werden kann.

Damit unterstützt er ein breiteres Spektrum von Einsatzmöglichkeiten.
Beispielsweise kann ein transformierter \enquote{Stream} übergeben werden, wie im Kapitel \ref{Schritt4} gezeigt.

Die einzige fehlende Komponente dafür ist ein \enquote{Stream},
der den zuletzt übermittelten Wert speichert und den neuen StreamBuilder Elementen übermittelt.
Deshalb wurde sich für das Package \enquote{rx.dart} entschieden,
welches genau dieses Verhalten mit dem \enquote{BehaviorSubject} abdeckt.
Durch dessen Verwendung kann sowohl auf das Registrieren des \IC{ChangeNotifierProvider} verzichtet werden
und es muss keine weitere Klasse für die einzelnen beobachtbaren Objekte erstellt werden.

Auch der \IC{MultiProvider} erscheint auf den ersten Blick als sehr nützlich.
Doch das Anbieten der Services durch ein eigens implementiertes \IC{InheritedWidget} erlaubt einen Zugriff,
der kürzer und expliziter ist.
Durch die Umstellung konnte der Zugriff auf das ViewModel mithilfe des Ausdrucks \IC{Provider.of<MassnahmenFormViewModel>(context, listen: false)} durch \IC{AppState.of(context).viewModel} ersetzt werden.

Eine ganz ähnliche, wenn auch deutlich kompliziertere Variante dieser Vorgehensweise, wurde auf der Google I/O 2018 von Filip Hracek und Matt Sulliivan vorgestellt. Doch anstatt lediglich das BehaviorSubject für das ViewModel zu verwenden, sorgte die Präsentation durch den zusätzlichen -- jedoch überflüssigen -- Einsatz zwei weiterer Stream-Klassen für schweres Verständnis \Lst{\ref{lst:CartBloc}}\footcite[][TC: \href{https://youtu.be/RS36gBEp8OI?t=1657}{27:37}]{GoogleIO18}.

\ifIncludeFigures
  \begin{listing}[h]
    \begin{minted}[]{dart}
class CartBloc{
    final _cart = Cart();

    Sink<Product> get addition => _additionalController.sink;

    final _additionController = StreamController<Product>();

    Stream<int> get itemCount => _itemCountSubject.stream;

    final _itemCountSubject = BehaviorSubject<int>();

    CartBloc(){
        _additionaController.stream.listen(_handle);
    }

    void _handle(Product product){
        _cart.add(product);
        _itemCountSubject.add(_cart.itemCount);
    }
}
\end{minted}
    \caption[Die Klasse CartBloc]{Die Klasse CartBloc, Quelle: \cite{GoogleIO18} TC: \href{https://youtu.be/RS36gBEp8OI?t=1657}{27:37} \footnotemark}
    \label{lst:CartBloc}
  \end{listing}
\fi

Obwohl das \IC{BehaviorSubject} die Funktionsweise des \IC{ViewModels} bereits löst,
wurde ein Objekt des Typs \IC{Sink} verwendet,
um Ereignisse von dem View an das \IC{ViewModel} senden zu können \Z{4}. \IC{StreamController} verwendet.
Ein Sind implementiert jedoch ausschließlich Methoden zum Hinzufügen von Ereignissen Punkt um den Stream zu lesen,
wird ein  dazugehöriger \IC{StreamController} erstellt \Z{6}.
Er hat im Gegensatz zum \IC{Sink} auch lesenden Zugriff auf die Ereignisse.
Sobald ein Ereignis eintrifft,
so wird es dem \IC{Model} \IC{_cart} hinzugefügt \Z{17}.
Es existiert außerdem ein weiterer \IC{Stream itemCount} \Z{8} welcher lediglich die transitive Eigenschaft der Anzahl der hinzugefügten Elemente darstellt \IC{18}. 
Er nutzt das BehaviorSubject \IC{10},
verwendet allerdings keine der bedeutsamen Methoden.
Es könnte genauso gut durch einen weiteren \IC{StreamController} ersetzt werden.

Der gesamte Quellcode kann stark vereinfacht werden \Lst{\ref{lst:CartBlocVereinfacht}}.
\ifIncludeFigures
  \begin{listing}[h]
    \begin{minted}[]{dart}
class CartBloc{
  final _cart = BehaviorSubject<Cart>(seedValue: Cart());

  addProduct(Product product) => _cart.value = _cart.value..add(product);

  Stream<int> get itemCount => _cart.map((cart) => cart.itemCount);
}
\end{minted}
    \caption[Die vereinfachte Klasse CartBloc]{Die vereinfachte Klasse CartBloc, Quelle: Eigenes Listing}
    \label{lst:CartBlocVereinfacht}
  \end{listing}
\fi

Durch Einsatz der für das \IC{BehaviorSubject} einzigartigen Getter-Methode \IC{value} kann dem \enquote{Stream} ein neues Objekt hinzugefügt werden,
wodurch er gleichzeitig ein neues Ereignis sendet \Z{4}.
Die Zuweisung hat zwar ansonsten keinen Zweck,
da das Objekt vor und nach der Zuweisung das gleiche ist,
denn es handelt sich um einen Referenztyp und nicht um einen Werttyp.
Die Erstellung weiterer \IC{StreamController} zum Senden der transitiven Eigenschaft \IC{itemCount} ist nicht nötig.
Sendet das \IC{BehaviorSubject} \IC{_cart} ein neues Event \Z{4},
so wird auch die Methode \IC{map} ausgelöst und ein transformiertes Eigenschaft gesendet \Z{6}.

Durch eine Anleitung mit diesem Ergebnis könnten gegebenenfalls weitere Entwickler das \enquote{BloC-Pattern} dem Paket \enquote{provider} vorziehen.

