

In dieser Arbeit wurde gezeigt, dass  das Hauptproblem der Formular Anwendung mithilfe von Funktionsobjekten und logischen Operatoren gelöst werden konnte. 

Auch die  Aktualisierung der sich tatsächlich ändernden Elemente in der Oberfläche wurde umgesetzt.
In jedem Fall war die deklarative und reaktive Programmierung der Oberfläche eine Erleichterung und Voraussetzung dafür.
Die Implementierung hätte auch mit \enquote{React Native} stattfinden können,
da es ebenso ein deklaratives Oberflächen Framework ist.
Die Stream Transformationen aus der Kern-Bibliothek von Dart und aus \enquote{RxDart} haben ihre Äquivalente in der Bibliothek \enquote{RxJS}.
\HP{https://www.learnrxjs.io/learn-rxjs/operators/filtering/distinctuntilchanged
}

Die Wahl von Flutter für die Entwicklung war trotz dessen aus den folgenden Gründen eine gute Entscheidung:

Die gesichteten Anleitungen für die Einarbeitung in das automatisierte Testen ebneten eine vollumfängliche und zielgerichtete Einarbeitung.
Keine weiteren Quellen von Drittanbietern mussten genutzt werden,
um die im Rahmen dieser Masterarbeit entstandenen \enquote{Unit-} und \enquote{Integrationstests} zu entwickeln.
Lediglich die initialen Probleme bei der Generierung von Mocks im Ordner für die Integrationstests stoppten die Entwicklung für einen Moment.

Hätte die Umsetzung in the? React Native stattgefunden,
so hätte die Einarbeitung in die Entwicklung von \enquote{Unit-} und \enquote{Integrationstest} eventuell einen höheren Aufwand bedeutet,
da die Dokumentation auf den unterschiedlichen Web-Portal in der Drittanbieter? verstreut ist.

Auch die Rezepte im Flutter-Kochbuch boten die benötigten Funktionalitäten wie die Formularvalidierung und
die Navigation über Routen. \HP{Mehr?}

Allerdings fällt die Wahl für das angemessene Zustandsmanagement für einen Anfänger in der deklarativen Programmierung nicht leicht.
Die Empfehlung von Google das Paket \enquote{Provider} zu nutzen, führte zu Schwierigkeiten,
wie in Sektion \ref{sec:Reevaluation-des-Zustandsmanagements}
beschrieben.
Das ursprünglich von Google beworbener \enquote{Bloc-pattern},
welches bei der \enquote{Flutter}-Community weniger beliebt ist,
war am Ende die angemessene Technologie.
Es fehlte aber die Dokumentation darüber,
wie es richtig eingesetzt wird.
Die Erkenntnisse,
die im Rahmen dieser Masterarbeit bezüglich der reibungslosen Implementierung des Zustandsmanagements mit \enquote{RxDart} gesammelt wurden,
sollen in Zukunft mit der \enquote{Flutter}-Community geteilt werden.

Das Wunschkriterium,
dem Benutzer auch die fehlerhafte Auswahl anzuzeigen,
die verhindert,
eine spezielle Option zu wählen,
konnte nicht umgesetzt werden.
Vor dem Hintergrund der für diese Arbeit festgelegten Ziele
und der Komplexität des Problems wurde sich gegen die Konzeption und Implementierung entschieden.
An den bisherigen Erkenntnissen soll jedoch weiter gearbeitet werden.
Nutzerumfragen sollen darüber hinaus zeigen,
in welcher Art und Weise eine solche Fehlermeldung präsentiert werden könnte.

 