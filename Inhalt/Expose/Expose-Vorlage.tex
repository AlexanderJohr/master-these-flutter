
\newpage
\thispagestyle{empty}

Hochschule Harz\newline
Fachbereich Automatisierung und Informatik
\vfill
\begin{center}

\large{\textsc{Thema und Aufgabenstellung der Masterarbeit}}

\large{\textsc{MA AI 29/2021}}

\vfill

\large{\textsc{für Herrn Alexander Johr}}

\vfill

%\vfill
\Large{\textsc{Entwicklung einer Formularanwendung mit Kompatibilitätsvalidierung der Einfach- und Mehrfachauswahl-Eingabefelder}}



\end{center}

\vfill

%gut darlegen könnendie methode spielt mehr die rolle


%andeuten nicht ins detail

%ich möchte mich testing


%kern des themas

%immer zurück kommen

%höhere eigen zum ergebis

%das liebesleben der tennisbällen unter einfluss des mondscheins

%größtmögliche unfall der passiert mal wahrscheinlichkeit 

%kompliziertheit komplexität

%werner von braun
%triebwerke geschrottet

%doku 
%arte 3 sat
%allgemein menschheit in den weltraum

%feindifferenziert


%Frage an Herr Ackermann / Singer: Vergleich mit Angular Dart, Angular allgemein, XAML WPF, evtl. etwas 

%Aber nicht alle

%Deshalb Fallstudie, richtig?

Das Thünen-Institut für Ländliche Räume wertet Daten zu Maßnahmen auf landwirtschaftlich genutzten Flächen aus. Dafür müssen entsprechende Maßnahmen
bundesweit mit Zeitbezug auswertbar sein und mit Attributen versehen werden.
Um die Eingabe für die Wissenschaftler des Instituts zu beschleunigen
 und um fehlerhafte Eingaben zu minimieren, soll eine 
 spezielle Formularanwendung entwickelt werden.
Neben herkömmlichen Freitextfeldern beinhaltet das gewünschte Formular zum Großteil Eingabefelder für Einfach- und Mehrfachauswahl.
Je nach Feld kann die Anzahl der Auswahloptionen mitunter zahlreich sein.
Dem Nutzer sollen daher nur solche Auswahloptionen angeboten werden,
die zusammen mit der zuvor getroffenen Auswahl sinnvoll sind.

\vspace{14pt}

Im Wesentlichen ergibt sich die Kompatibilität 
der Auswahloptionen aus der Bedingung, 
dass für dasselbe oder ein anderes Eingabefeld eine Auswahlmöglichkeit gewählt bzw.
nicht gewählt wurde. Diese Bedingungen müssen durch 
Konjunktion und Disjunktion verknüpft werden können.
In Sonderfällen muss ein Formularfeld jedoch auch 
die Konfiguration einer vom Standard abweichenden Bedingung
ermöglichen. 
Wird dennoch versucht,
eine deaktivierte Option zu selektieren, wäre eine Anzeige der
inkompatiblen sowie der stattdessen notwendigen Auswahl ideal.

\vspace{14pt}
Die primäre Zielplattform der Anwendung ist das Desktop-Betriebssystem
Microsoft Windows 10.
Idealerweise ist die Formularanwendung auch auf weiteren Desktop-Plattformen sowie
mobilen Endgeräten wie Android- und iOS-Smartphones und -Tablets
lauffähig. Die Serialisierung der eingegebenen Daten genügt dem Institut 
zunächst als Ablage einer lokalen Datei im JSON-Format. 


\vspace{14pt}
Die Masterarbeit umfasst folgende Teilaufgaben:
\begin{itemize}
    \itemsep0em
\item Analyse der Anforderungen an die Formularanwendung
\item Evaluation der angemessenen Technologie für die Implementierung
\item Entwurf und Umsetzung der Übersichts- und Eingabeoberfläche
\item Konzeption und Implementierung der Validierung der Eingabefelder
\item Entwicklung von automatisierten Testfällen zur Qualitätskontrolle
\item Bewertung der Implementierung und Vergleich mit den Wunschkriterien
\end{itemize}


%Die Freiheiten der Oberflächenentwicklung in Flutter sollen genutzt werden, um 
%visuellen Komponenten zu erstellen, die durch Einsatz des Strategie-Entwurfsmusters
%im hohen Maße anpassbar sind.



%Da die Aktualisierung von Flutter-Views nicht automatisch erfolgt, muss ein angemessenes 
%Zustandsmanagement evaluiert und eingesetzt werden.
%Eine angemessene Technologie zur Serialisierung soll gewählt und 
%das Persistieren der eingegebenen Datensätze damit umgesetzt werden.
%Durch die Menge der Auswahlfelder ist bei Weiterentwicklung der APIs mit 
%hohem Migrationsaufwand der Codebasis zu rechnen. Der Einsatz von regulären Ausdrücken
%soll helfen, den Prozess zu automatisieren. Die Entwicklung automatisierter Testfälle
%soll weiterhin ermöglichen, Fehler bei der Weiterentwicklung zeitnah zu erkennen.



%Die Masterarbeit beinhaltet damit folgende Teilaufgaben:
%\begin{itemize}
%	\itemsep0em
%	\item Evaluation des Zustandsmanagements und Implementierung des View Models
%    \item Entwicklung von wiederverwendbarer Komponenten des Views und Anpassung dieser mit dem Strategie Entwurfsmuster
%    \item Auswahl der Technologie zur Serialisierung und Implementierung der Persistierung des Models
%	\item Migration der Codebasis auf aktualisierte APIs mittels Regulärer Ausdrücke
%	\item Entwicklung von Automatisierten Testfällen
%\end{itemize}

\vfill
\vfill
\vfill 
\vfill
\vfill

\begin{tabularx}{\textwidth}{@{} *2{>{\centering\arraybackslash}X}@{}}
Prof. Jürgen Singer Ph.D. & Prof. Daniel Ackermann \\
1. Prüfer                 & 2. Prüfer	 \\
\end{tabularx}	     

