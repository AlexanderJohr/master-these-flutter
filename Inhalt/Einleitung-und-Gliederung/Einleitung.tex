\chapter{Einleitung}

Eine angenehme Erfahrung für den Nutzer einer Software entsteht unter anderem dann,
wenn ihm die richtigen Informationen zur richtigen Zeit präsentiert werden.

In Formularen spielen Einfach- und Mehrfachauswahlfelder
-- im Englischen unter dem Begriff \enquote{multiple choice} zusammengefasst -- eine Rolle.

%Abbildung x zeigt eine solches Auswahlfeld mit 3 Auswahloptionen.

% Zieleinheit
% \newlist{todolist}{itemize}{2}
% \setlist[todolist]{label=$\square$}
% \begin{center}
%   \begin{todolist}[topsep=0pt,itemsep=-1ex,partopsep=1ex,parsep=1ex]
%     \item Stück
%     \item m³
%     \item ha
%   \end{todolist}    
% \end{center}

Die richtigen Informationen zur richtigen Zeit zu präsentieren, könnte in diesem Kontext bedeuten,
nur solche Auswahloptionen anzubieten,
welche mit den bisherigen gewählten Optionen Sinn ergeben.
Für die Datenerfassung von Maßnahmen auf landwirtschaftlich genutzten Flächen stellt dies eine Herausforderung dar,
denn die Auswahlfelder und Optionen sind zahlreich und ihre Bedingungen komplex.
Es lassen sich folgende Probleme ableiten. 

\section{Problemstellung}

Das primäre Problem und damit Musskriterium der Formularanwendung ist,
dass sich die Auswahlfelder untereinander beeinflussen.
Wird eine Option in einem Auswahlfeld selektiert,
so werden die möglichen Auswahlfelder von potenziell jedem weiteren Auswahlfeld dadurch manipuliert.
Es muss eine Möglichkeit gefunden werden,
die Abhängigkeiten in einer einfachen Art und Weise für jede Auswahloption zu hinterlegen und bei Bedarf abzurufen.


Das sekundäre Problem,
welches sich vom primären Problem ableiten lässt,
ist die  Laufzeitgeschwindigkeit. 
Wenn die Auswahl in einem Auswahlfeld die Auswahlmöglichkeiten in potenziell allen anderen Auswahlfeldern manipuliert,
so könnte dies zu einer hohen Last beim erneuten Zeichnen der Oberfläche zur Folge haben.
Wann immer der Nutzer eine Selektion tätigt,
müsste das gesamte Formular neu gezeichnet werden,
um sicherzustellen,
dass invalide Auswahloptionen gekennzeichnet werden.
Bei einem  Formular mit wenigen Auswahlfeldern wäre das kein Problem,
doch die nötigen Auswahlfelder für das Eintragen von Maßnahmen des Europäischen Landwirtschaftsfonds für die Entwicklung des ländlichen Raums (\enquote{ELER}) sind zahlreich.
Ein  automatisierter Integrationstest,
welcher im Formular Daten einer beispielhaften Maßnahme einträgt,
zählt zum Zeitpunkt der Erstellung dieser Arbeit bereits 58 aufgerufene Auswahlfelder und 107 darin selektierte Auswahloptionen.
Das bedeutet,
dass bei jedem dieser 107 Selektionen die 58 Auswahlfelder und all ihre Kinder neu gezeichnet werden müssten.
Es entstehen also Wartezeiten nach jedem Auswählen einer Option.
Das Formular soll in Zukunft zudem noch erweitert und auch für die Eingabe ganz anderer Datensätze mit potenziell noch mehr Auswahlfeldern eingesetzt werden können.
Die Dateneingabe wäre mit den Wartezeiten trotzdem möglich.
\begin{newText}Daher ist es ein Wunschkriterium, dass ein Mechanismus gefunden wird,
der nur die Elemente neu zeichnet, welche sich wirklich ändern.\end{newText}

Ein weiteres Wunschkriterium ist,
dass der Benutzer beim Anwählen einer deaktivierten Auswahloption eine Mitteilung darüber erhält,
welche der zuvor ausgewählten Optionen zu der Inkompatibilität mit den gewünschten Optionen führt.

Ziel dieser Masterarbeit ist es, eine geeignete Technologie für die Umsetzung auszuwählen
und die Umsetzbarkeit der  oben genannten Kriterien zu evaluieren.

\section{Gliederung}






Kapitel \ref{chap:Technologie-Auswahl} evaluiert die Kandidaten der Frontend-Technologien,
die für eine nähere Betrachtung infrage kommen.
Dazu werden die Umfrageergebnisse der \enquote{Stack Overflow}-Umfragen sowie das relative Suchinteresse dieser Technologien auf Google Trends analysiert.
Da die Technologien \enquote{React Native} und  \enquote{Flutter} als die am verbreitetsten Technologien hervorgingen,
werden sie daraufhin einem detaillierteren Vergleich unterzogen.

Da als Frontend-Technologie für die Entwicklung der Formularanwendung \enquote{Flutter} gewählt wurde,
beschäftigt sich Kapitel \ref{chap:Grundlagen} mit den Grundlagen des Frameworks und der zugrundeliegenden Programmiersprache \enquote{Dart}.

Die Kapitel \ref{chap:Schritt-1} bis \ref{chap:Schritt-7} dokumentieren die nötigen Entwicklungsschritte,
um die einzelnen aufeinander aufbauenden  Funktionalitäten hinzuzufügen.
Die während der Arbeit im Thünen-Institut entstandene Anwendung wurde zu diesem Zweck auf die für die  Problemstellung bedeutsamsten Funktionalitäten reduziert.
Die Anzahl der Auswahlfelder beschränkt sich darüber hinaus auf ein Mindestmaß,
welches die Bedingungen der Auswahloptionen untereinander erkennbar macht. 

In Kapitel \ref{chap:Schritt-1} wird die grundlegende Struktur der Anwendung hergestellt.
Kapitel \ref{chap:Schritt-2} fügt Hilfsmethoden hinzu,
welche  das Hinzufügen  weiterer Formularfelder  in den folgenden Schritten vereinfachen wird.

In Kapitel \ref{chap:Schritt-3} erhält die Anwendung die grundlegende Funktion,
Felder zu validieren.
Kapitel \ref{chap:Schritt-4} erweitert die Validierung schließlich um die Bedingungen der Auswahloptionen.
Als Konsequenz werden alle Formularfelder neu gezeichnet,
sollte  der Benutzer eine beliebige Auswahloption selektieren.
Durch die Validierung  geschieht es nach dem Neuzeichnen,
dass invalide Auswahlfelder rot markiert werden.
Die erforderlichen Änderungen,
um nur die Auswahlfelder zu aktualisieren,
die ihre Validität oder ihren eigenen Inhalt ändern,
werden in Kapitel \ref{chap:Schritt-5} hinzugefügt.

Kapitel \ref{chap:Schritt-6} ergänzt die Möglichkeit,
Mehrfachauswahlfelder zu verwenden.
Kapitel \ref{chap:Schritt-7} sorgt dafür,
dass auch benutzerdefinierte Bedingungen für die Auswahlfelder hinterlegt werden können.

Die während der Entwicklung der Anwendung gesammelten Erkenntnisse werden in Kapitel \ref{chap:Diskussion} dargelegt.
Kapitel \ref{chap:Schlussfolgerung-und-Ausblick} bewertet die Erkenntnisse,
ergänzt sie um einen Ausblick und vergleicht die Ergebnisse der Entwicklung mit den Anforderungen.
