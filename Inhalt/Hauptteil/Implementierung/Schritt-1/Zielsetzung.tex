\chapter{Schritt 1 -- Formular in Grundstruktur erstellen}
\label{chap:Schritt-1}

Im ersten Schritt soll die Formularanwendung in ihrer Grundstruktur entwickelt werden.
 Das beinhaltet alle drei Oberflächen, welche in den darauf folgenden Schritten lediglich erweitert werden.
 Das Formular erhält noch keine  Validierung.
 Somit sind alle Eingaben oder nicht kompatible Selektionen erlaubt.
 Die erste Ansicht, welche der Benutzer sieht, soll die Übersicht der bereits eingetragenen Maßnahmen sein \Abb{\ref{fig:Schritt1Uebersicht}}.
 
% TODO: rausgekürzt, doch wieder rein nehmen?
%Dort ist auch zu sehen, dass die Anwendung ohne Anpassungen zunächst einmal im sogenannten Material Design\footnoteI{Material Design umfasst eine Reihe von Prinzipien zur Auszeichnung von Benutzeroberflächen.Das ist Design-System wurde von Google Inc. entwickelt. Der Name leitet sich daher ab, dass Objekte mit der Nachahmung physikalischer Eigenschaften -- wie etwa dem Werfen eines Schattens -- den Eindruck von tatsächlichen Materialien erwecken.\footnote{\cite{MaterialDesignIntroduction}}} gestylt ist.
\begin{alexfigure}{Inhalt/Hauptteil/Implementierung/Schritt-1/Ue.png}
  {Der Übersichtsbildschirm in Schritt 1}
  {Der Übersichtsbildschirm in Schritt 1}

  \label{fig:Schritt1Uebersicht}

\end{alexfigure}

Die Auflistung der Maßnahmen erfolgt in den Kategorien \enquote{In Bearbeitung} und \enquote{Abgeschlossen}.
Innerhalb dieser Rubriken werden die Maßnahmen in einer Tabelle angezeigt.
Mit einem Klick auf den Button unten rechts im Bild wird der Benutzer auf die zweite Ansicht weitergeleitet: die Eingabemaske \Abb{\ref{fig:Schritt1Eingabemaske}}.

\begin{alexfigure}{Inhalt/Hauptteil/Implementierung/Schritt-1/D.png}
  {Die Eingabemaske in Schritt 1}
  {Die Eingabemaske in Schritt 1}

  \label{fig:Schritt1Eingabemaske}

\end{alexfigure}

Sie ermöglicht die Eingabe des Maßnahmentitel über ein simples Eingabefeld.
Außerdem ist die Selektionskarte für den Status zu sehen.
Mit einem Klick auf diese Karte öffnet sich der Selektionsbildschirm.
Er ermöglicht die Auswahl der Auswahloptionen, in diesem Fall die Optionen \enquote{in Bearbeitung} und \enquote{abgeschlossen}
\Abb{\ref{fig:Schritt1SelektionsBildschirmStatus}}.

\begin{alexfigure}{Inhalt/Hauptteil/Implementierung/Schritt-1/D D.png}
  {Der Selektionsbildschirm in Schritt 1}
  {Der Selektionsbildschirm in Schritt 1}

  \label{fig:Schritt1SelektionsBildschirmStatus}

\end{alexfigure}
