
\ifodd\value{page}\hbox{}\newpage\fi
\section{Das \enquote{MassnahmenTable}-\enquote{Widget}}

Die \IC{MassnahmenTable} ist ein  \IC{StatelessWidget} \LstZ{\ref{lst:Schritt1KlasseMassnahmenTable}}{6}.
Zur Anzeige eignet sich das \enquote{Widget} \IC{Table} \Z{15-31}. Im Verlauf der Erstellung der Arbeit wurde versucht, das \enquote{Widget} \IC{DataTable} zu verwenden.
Doch im Gegensatz zur \IC{DataTable} erlaubt es das \enquote{Widget} \IC{Table}, unterschiedlich hohe Zeilen zu zeichnen.
Die Höhe der Zeile wird dazu in Abhängigkeit von dem benötigten Platz des Inhalts der Zellen berechnet.\pdfcomment[icon=Note,color=yellow]{29.08.2021 Satz umformuliert}
Die Breite und Ausrichtung der Spalten kann konfiguriert werden.
Die Eigenschaft \IC{IntrinsicColumnWidth} sorgt dafür,
dass die Spalten immer genau so groß sind,
wie der Inhalt es benötigt \Z{17}. 
Zeilenumbrüche für die Texte in den Spalten sind somit nicht notwendig.
\IC{TableCellVerticalAlignment.middle} lässt die  Tabelle die Inhalte zentriert darstellen \Z{18}.


\begin{alexlisting}{Schritt 1}{Die Klasse \enquote{MassnahmenTable}}
    {Quellcode/Schritt-1/conditional_form/lib/widgets/massnahmen_table.dart}
    {firstline=4, lastline=32}
    \label{lst:Schritt1KlasseMassnahmenTable}
  \end{alexlisting}

Der Parameter \IC{children} erhält als Argument eine Liste von \IC{TableRow}-Elementen \Z{19-30}.
Die erste Tabellenzeile beinhaltet die Spaltenbezeichnungen \Z{20-23}.
Jede \IC{TableRow} hat wiederum den Parameter \IC{children}.
Das Argument bezieht sich hier auf die Zellen in der Zeile.
Dabei ist wichtig, dass jede \IC{TableRow} die gleiche Anzahl von Zellen hat.
Weicht nur eine Zeile davon ab, zeichnet sich die gesamte Tabelle nicht und eine Ausnahme wird ausgelöst.

\clearpage
Nach den Spaltenbezeichnungen folgen die Zeilen für die Daten der Maßnahmen \Z{24-29}.
Die Methode \IC{map} \Z{24} ermöglicht es dazu, durch die Liste der Maßnahmen zu iterieren und für jede Maßnahme ein Element eines völlig anderen Typs -- in diesem Fall \IC{TableRow} -- zurückzugeben.
Bei den vorangestellten Punkten \IC{...} in Zeile 24 handelt es sich um den \enquote{spread operator}.\pdfcomment[icon=Note,color=yellow]{29.08.2021 Fehler: es hat gefehlt}
Die Filtermethode \IC{map} und die darauffolgende Methode \IC{toList} liefern eine Liste von \IC{TableRow}-Elementen.
Die umgebende Liste der Zeilen \IC{children} \Z{19-30} erwartet jedoch Elemente des Typs \IC{TableRow} und keine Elemente des Typs \IC{List}.
Der \enquote{spread operator} ermöglicht,
alle Elemente der inneren Liste in die äußere Liste einzufügen. \footcite[Vgl.][]{SpreadOperator}



  Für die Spaltenbezeichnungen wurde eine Hilfsmethode kreiert: \IC{_buildColumnHeader} \Lst{\ref{lst:Schritt1buildColumnHeader}}.
  Sie zeichnet die Spalten mit einem Abstand von \IC{8} Pixeln in alle Richtungen.

  \begin{alexlisting}{Schritt 1}{Die Hilfsmethode \enquote{_buildColumnHeader}}
    {Quellcode/Schritt-1/conditional_form/lib/widgets/massnahmen_table.dart}
    {firstline=34, lastline=37}
    \label{lst:Schritt1buildColumnHeader}
  \end{alexlisting}

  Eine weitere Hilfsmethode \IC{_buildSelectableCell} erstellt Zellen, die anklickbar sind \Lst{\ref{lst:Schritt1buildSelectableCell}}.
  Das \enquote{Widget} \IC{TableRowInkWell} \Z{41-51} kann in Tabellen verwendet werden, um einen anklickbaren Bereich zu erstellen.
  Beim  Anklicken breitet sich ausgehend von der Position des Klicks ein Tintenklecks  aus.
  Dabei überschreitet der Tintenklecks nicht den Bereich, der von der umgebenden Zeile begrenzt ist.
  Beim Auslösen des Ereignisses \IC{onTap} erfolgt die Ausführung der Rückruffunktion \IC{onSelect} \Z{44} mit der ausgewählten Maßnahme.
  Doch zuvor muss überprüft werden,
  ob die Rückruffunktion auch initialisiert wurde \Z{43}.
  Wie hier zu sehen ist, reicht es nicht aus, abzufragen, ob \IC{onSelect}  gesetzt ist.
  Es erfolgt keine Typ-Beförderung zu einem Typ ohne Null-Zulässigkeit, denn es handelt sich um eine Instanzvariable.
  Deshalb muss das Suffix \IC{!} gesetzt sein \Z{44} (Siehe Grundlagenkapitel \ref{sec:TypenMitNullZulaessigkeit} \enquote{\nameref{sec:TypenMitNullZulaessigkeit}} auf Seite \pageref{sec:TypenMitNullZulaessigkeit}).
  
  \begin{alexlisting}{Schritt 1}{Die Hilfsmethode \enquote{_buildSelectableCell}}
    {Quellcode/Schritt-1/conditional_form/lib/widgets/massnahmen_table.dart}
    {firstline=39, lastline=51}
    \label{lst:Schritt1buildSelectableCell}
  \end{alexlisting}

  \clearpage
  Bei \IC{onSelect} handelt es sich um eine Rückruffunktion.
  An diesem Beispiel kann das \enquote{inversion of control pattern} visualisiert werden.
  Abbildung \ref{lst:UmlOnSelet} zeigt, wie die Akteure zusammenarbeiten.\pdfcomment[icon=Note,color=yellow]{29.08.2021 Fehler komma hat gefehlt}
  Der \enquote{MassnahmenMasterScreen} verwendet die \enquote{MassnahmenTable}.
  Die Tabelle enthält ein Objekt namens \enquote{onSelect}.
  Dabei handelt es sich um ein Funktionsobjekt.
  Anstatt eine neue Klasse mit einer beinhaltenden Funktion zu deklarieren, kann das Gleiche über eine Abkürzung erreicht werden:
  dem Schlüsselwort \IC{typedef} \Lst{\ref{lst:Schritt1TypedefOnSelectCallback}}.
  
  \begin{alexlisting}{Schritt 1}{Die Klasse \enquote{MassnahmenTable}}
    {Quellcode/Schritt-1/conditional_form/lib/widgets/massnahmen_table.dart}
    {firstline=4, lastline=4}
    \label{lst:Schritt1TypedefOnSelectCallback}
  \end{alexlisting}
  
  Hier erlaubt es eine Funktionssignatur als einen Typ zu deklarieren.
  Der \enquote{MassnahmenMasterScreen} wiederum instanziiert ein anonymes Funktionsobjekt,
  welches der Schnittstelle -- und damit der Funktionssignatur -- entspricht.



  Weil es der Signatur der Typdefinition von \IC{OnSelectCallback} entspricht,
  kann es der Tabelle als Argument für den Parameter \IC{onSelect} übergeben werden \LstZ{\ref{lst:Schritt1MassnahmenTableOnSelect}}{75-82}. 
  

  



Das \enquote{inversion of control pattern} ist auch unter dem Namen \enquote{hollywood pattern} bekannt,
da es ähnlich wie die typische Antwort auf eine Bewerbung für einen Hollywood Film
-- don’t call us, we’ll call you --
funktioniert.\footcite[Vgl.][]{fowler2005InversionOfControl}

Genauso arbeiten der Übersichtsbildschirm und die Tabelle zusammen.
Der Übersichtsbildschirm verwendet die Tabelle, welche nicht wissen muss, wofür sie eingesetzt wird.
Sobald die Tabelle eine Selektion des Benutzers bemerkt, kommuniziert sie wieder mit dem Übersichtsbildschirm, worauf dieser über den \enquote{Service Locator} auf das \enquote{ViewModel} zugreift, um die selektierte Maßnahme zu übergeben.

\clearpage


\begin{alexlisting}{Schritt 1}{Die Ausgabe der Maßnahmen}
    {Quellcode/Schritt-1/conditional_form/lib/screens/massnahmen_master.dart}
    {firstline=70, lastline=82}
    \label{lst:Schritt1MassnahmenTableOnSelect}
  \end{alexlisting}

\vspace{21mm}

\ifIncludeFigures
  \begin{figure}[h]
    \centering

    \begin{tikzpicture}
      \begin{umlseqdiag}
    
        \umlsimpleclass[x=0, y=-2.7]{MassnahmenMasterScreen}{
        }{
        }
    
        \umlsimpleclass[x=8,y=-2.7]{MassnahmenTable}{
        }{
        }
    
        \umlsimpleclass[x=8.0,y=0,type=typedef]{OnSelectCallback}{
        }{
        }
    
    
        \umlobject[x=0,y=0, class=OnSelectCallback]{onSelect}{
        }{
        }
    
    
    
    
        \umldep[arg=<<instance of>>, pos=0.5]{onSelect}{OnSelectCallback}
        \umldep[arg=<<instantiate>>, pos=0.5]{MassnahmenMasterScreen}{onSelect}
    
    
        \umluniassoc[]{MassnahmenMasterScreen}{MassnahmenTable}
        \umluniaggreg[arg=onSelect, pos=0.5]{MassnahmenTable}{OnSelectCallback}
      \end{umlseqdiag}
    \end{tikzpicture}

    \caption[\enquote{UML}-Diagramm des \enquote{inversion of control pattern} für \enquote{MassnahmenMasterScreen} und \enquote{MassnahmenTable}]
    {\enquote{UML}-Diagramm des \enquote{inversion of control pattern} für \enquote{MassnahmenMasterScreen} und \enquote{MassnahmenTable}, Quelle: Eigene Abbildung}
    \label{lst:UmlOnSelet}

  \end{figure}%
\fi


