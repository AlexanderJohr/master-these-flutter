
\ifodd\value{page}\hbox{}\newpage\fi
\subsection{Speicher-Routine}


Die Funktion, die dem Parameter \IC{onWillPop} des \IC{WillPopScope} übergeben wurde, ist in Listing \ref{lst:Schritt1SaveRecord} zu sehen.
Die Voraussetzung für diese Funktion ist, dass ihr Rückgabetyp ein \IC{Future<bool>} ist.
Das erlaubt der Methode, asynchron zu sein.
Der \IC{Future}, der von der Funktion zurückgegeben werden soll, muss in der Zukunft den Wert \IC{true} zurückgeben, wenn dem \enquote{Navigator} erlaubt  werden soll, zurück zu navigieren.
Da die Implementierung der Methode allerdings nicht asynchron ist, soll der Wahrheitswert direkt zurückgegeben werden.
Mit dem benannten Konstruktor \IC{value} der Klasse \IC{Future} ist es möglich, genau das zu tun \Z{27}.
 Der Wahrheitswert ist damit in einem \IC{Future}-Objekt gekapselt und steht ohne Verzögerung zur Verfügung.
Aktuell soll die Maßnahme lediglich abgespeichert werden \Z{25}, da noch keine Validierung erfolgt.

Der Benutzer erhält noch eine Mitteilung,
dass die Maßnahme erstellt wurde.
Das aktuelle \IC{Scaffold}-Objekt kann über \IC{ScaffoldMessenger.of} adressiert werden \Z{20}. Sollte bereits eine Mitteilung vorliegen,
wird diese wieder versteckt,
um Platz für die neue zu schaffen \Z{21}. Anschließend wird eine sogenannte \IC{Snackbar} mit dem entsprechenden Text angezeigt \Z{22-23}.

\begin{alexlisting}{Schritt 1}{Die Funktion saveRecordAndGoBackToOverviewScreen}
  {Quellcode/Schritt-1/conditional_form/lib/screens/massnahmen_detail/massnahmen_detail.dart}
  {firstline=19, lastline=28}
  \label{lst:Schritt1SaveRecord}
\end{alexlisting}



\section{Das \enquote{SelectionCard}-\enquote{Widget}}

Das Listing \ref{lst:Schritt1KlasseSelectionCard} zeigt die Struktur des \enquote{Widgets} \IC{SelectionCard}.
Die Klasse hat einen generischen Typparameter \Z{15}.
\IC{<ChoiceType extends Choice>} bedeutet, dass die \IC{SelectionCard} nur für Typen verwendet werden kann, die von \IC{Choice} erben.
Das ist eine wichtige Voraussetzung, da auf den übergebenen Werten Operationen ausgeführt werden sollen, die nur \IC{Choice} unterstützt.
Alle Instanzvariablen werden über diesen Typparameter generalisiert \Z{17-20}, mit Ausnahme der Instanzvariablen \IC{title}, denn sie erhält kein Typargument \Z{16}.

\begin{alexlisting}{Schritt 1}{Die Klasse SelectionCard}
  {Quellcode/Schritt-1/conditional_form/lib/widgets/selection_card.dart}
  {firstline=7, lastline=31}
  \label{lst:Schritt1KlasseSelectionCard}
\end{alexlisting}

Mit dem \enquote{Stream} \IC{selectionViewModel} verwaltet die \IC{SelectionCard} ihren eigenen Zustand.
Der \enquote{Stream} ist mit dem generischen Typ \IC{BuiltSet<ChoiceType>} konfiguriert.
Das macht es unmöglich, den aktuell hinterlegten Wert anzupassen, ohne das Gesamtobjekt auszutauschen.
Der Tausch des Objektes wiederum bewirkt, dass ein Ereignis über den \enquote{Stream} ausgelöst wird. Über dieses Ereignis zeichnet die SelectionCard Teile seiner Oberfläche neu.
Allerdings erhält der Konstruktor kein Argument des Typs \IC{BehaviorSubject}, sondern stattdessen vom \IC{Iterable<ChoiceType>} \Z{24}.
Damit wird der Benutzer nicht darauf eingeschränkt, einen \enquote{Stream} zu übergeben.
Er kann auch eine gewöhnliche Liste oder Menge setzen.
Die Umwandlung der ankommenden Kollektion erfolgt in der Initialisierungsliste \Z{29-30}.
Nur so ist es möglich, die Instanzvariable mit \IC{final} als unveränderbar zu kennzeichnen.
Initialisierungen solcher Variablen müssen im statischen Kontext der Objekterstellung geschehen.
Der Konstruktor-Körper gehört dagegen nicht mehr zum statischen Teil.
Im Konstruktor-Körper können Operationen der Instanz verwendet werden, denn das Objekt existiert bereits.
Der Versuch eine mit \IC{final} gekennzeichnete Instanzvariable im Konstruktor-Körper zu setzen, führt zu einem Compilerfehler in \enquote{Dart}.
Der Konstruktor \IC{seeded} der Klasse \IC{BehaviorSubject} wird mit einem \IC{BuiltSet} gefüllt \Z{29}.
Dieses wiederum wird mit dem benannten Konstruktor \IC{from} von \IC{BuiltSet} mit der Kollektion aufgerufen \Z{30}.
Er wandelt die  Liste in eine unveränderbare Menge um.
Die Liste aller Auswahloptionen \IC{allChoices} \Z{18} gewährleistet über den generischen Typparameter, dass nicht versehentlich Auswahloptionen übergeben werden, die nicht zum Typ der \IC{SelectionCard} passen.
Die Rückruffunktionen \Z{19, 20}, die bei Selektion und Deselektion von Optionen ausgelöst werden, bieten einen besonderen Vorteil dadurch, dass sie mit dem generischen Typ konfiguriert sind.
Die Signaturen der Rückruffunktionen \Z{7-8, 10-11} geben nämlich vor, dass der erste Parameter vom Typ \IC{ChoiceType} sein muss.
Wenn nun der Benutzer der \IC{SelectionCard} einen Typ wie etwa \IC{LetzterStatus} für den Typparameter übergibt, so erhält er auch eine Rückruffunktion, deren erster Parameter vom Typ \IC{LetzterStatus} ist.
Ohne eine Typumwandlung -- englisch \enquote{type casting} -- von \IC{Choice} in \IC{LetzterStatus}, können keine Operationen auf dem Objekt angewendet werden, die nur die Klasse \IC{LetzterStatus} unterstützt.

Das erste Element, welches von der \IC{build}-Methode zurückgeben wird, ist ein \IC{StreamBuilder} \LstZ{\ref{lst:Schritt1BuildMethodeDerSelectionCard}}{47}.
Er horcht auf das \IC{selectionViewModel} \Z{48}.
Sobald also eine Selektion getätigt wurde, aktualisiert sich auch die dazugehörige Karte.
Das Aussehen einer Karte wird durch das \enquote{Widget} \IC{Card} erzielt \Z{51}.
Dadurch erhält es abgerundete Ecken und einen Schlagschatten, der es vom Hintergrund abgrenzt.
Ein \IC{ListTile}-\enquote{Widget} erlaubt es dann, den übergebenen \IC{titel} als Überschrift zu setzen \Z{54} und die aktuell ausgewählten Selektionen als Untertitel anzuzeigen \Z{56}.
Zu diesem Zweck wandelt die Methode \IC{map} alle Elemente von \IC{selectedChoices} in \IC{String}-Objekte um, indem es von dem \IC{Choice}-Objekt lediglich den Beschreibungstext \IC{description} verwendet.
Anschließend sammelt der Befehl \IC{join} die resultierenden \IC{String}-Objekte ein, formt sie in einem gemeinsamen \IC{String} zusammen und trennt sie darin jeweils mit einem \IC{","} voneinander.



Das \IC{ListTile} erhält ein \IC{FocusNode}-Objekt \Z{53}, damit der Benutzer beim Zurücknavigieren von der Unterseite im Formular wieder in der gleichen vertikalen Position der Karte landet, die er zuvor ausgewählt hat.
Der Benutzer würde ansonsten im Formular wieder an der obersten Position ankommen.
Der \IC{FocusNode} wird einmal zu Anfang der \IC{build}-Methode erstellt \Z{35}.
Damit ist er außerhalb der Methode \IC{builder} des \IC{StreamBuilder}-\enquote{Widgets} und bleibt somit beim Neuzeichnen der Karte erhalten.


Klickt der Benutzer die Karte an, navigiert er schließlich zur Unterseite, wo er die Auswahloptionen präsentiert bekommt.
Die verschachtelte Funktion \IC{navigateToSelectionScreen} kommt dafür zum Einsatz \Z{37-45}.
Da das Wechseln zur Unterseite bevorsteht, fordert der \IC{focusNode} den Fokus für das angeklickte \IC{ListTile} an \Z{38}.
Schließlich navigiert der Benutzer mit \IC{Navigator.push} zur Unterseite.
Es handelt sich um den Selektionsbildschirm, auf dem der Benutzer die gewünschte Option anwählen kann.
Die Besonderheit dieses Mal ist: Die Route ist nicht als \enquote{Widget} deklariert und wird nicht über einen Namen aufgerufen, so wie es bei dem Übersichtsbildschirm und der Eingabemaske war.
Stattdessen baut eine Funktion bei jedem Aufruf die Seite neu.
Das dynamische Bauen der Seite hat einen besonderen Vorteil, der am Listing \ref{lst:Schritt1FunktionCreateMultipleChoiceSelectionScreen} erklärt wird.


\clearpage

\begin{alexlisting}{Schritt 1}{Die \enquote{build}-Methode der Klasse \enquote{SelectionCard} in Schritt 1}
  {Quellcode/Schritt-1/conditional_form/lib/widgets/selection_card.dart}
  {firstline=34, lastline=62}
  \label{lst:Schritt1BuildMethodeDerSelectionCard}
\end{alexlisting}


\subsection{Bildschirm für die Auswahl der Optionen}

 
Die Methode \IC{createMultipleChoiceSelectionScreen} \LstS{lst:Schritt1FunktionCreateMultipleChoiceSelectionScreen} gibt einen \IC{Scaffold} zurück,
der die gesamte Seite enthält \Z{65}.
Das erste Kind des \IC{Scaffold} ist wiederum ein \IC{StreamBuilder} \Z{69}.
Hier wird der Vorteil der dynamischen Erzeugung der Seite offensichtlich: die Unterseite kann das gleiche \enquote{ViewModel} wiederverwenden, welches auch von der \IC{SelectionCard} genutzt wird.
Auch alle weiteren Instanzvariablen der \IC{SelectionCard} können wiederverwendet werden.
Würde es sich stattdessen um eine weitere Route handeln, so müssten alle diese Informationen über den \enquote{Navigator} zur neuen Unterseite übergeben werden.
Sollte der Nutzer die Auswahl beenden, so müsste auch ein Mechanismus für das Zurückgeben der selektierten Daten implementiert werden.
Dadurch, dass die \IC{SelectionCard} und der Selektionsbildschirm sich das gleiche \enquote{ViewModel} teilen,
kann sogar ein weiterer Vorteil in Zukunft genutzt werden:
In einem zweispaltigen Layout könnte auf der linken Seite die Eingabemaske und auf der rechten Seite der Bildschirm der Auswahloptionen eingeblendet werden.
Sobald sich Auswahloptionen im rechten Selektionsbildschirm verändern, so würden sich die Änderungen auf der linken Seite für den Benutzer direkt widerspiegeln.

Innerhalb des \IC{StreamBuilder} werden die Auswahloptionen gebaut.
Dazu speichert die lokale Variable \IC{selectedChoices} die aktuellen Selektionen des \enquote{Streams} zunächst zwischen \Z{72}.
Die Optionen werden in einem ListView präsentiert \Z{73}.
Er ermöglicht es, Listen-Elemente in einem vertikalen Scrollbereich darzustellen.
Die Funktion \IC{map} konvertiert alle Objekte in der Liste aller möglichen Optionen \IC{choices} in Elemente des Typs \IC{CheckboxListTile} \Z{74-98}.
In der Standard-Variante sind die Checkboxen rechtsbündig.
Der Parameter \IC{controlAffinity} kann genutzt werden, um dieses Verhalten zu überschreiben \Z{80}.

Das \IC{CheckboxListTile} erhält einen Titel, der aus dem Beschreibungstext \IC{description} des \IC{Choice}-Objektes gebildet wird \Z{81}.
Ob eine Option aktuell bereits ausgewählt ist, kann mit dem Parameter \IC{value} gekennzeichnet werden \Z{82}.
Sollte sich die Selektion ändern, erfolgt die Mitteilung über die Rückruffunktion \IC{onChanged} \Z{83-94}.
Der erste Parameter der anonymen Funktion gibt dabei die ausgewählte Selektion an.
Eine Fallunterscheidung überprüft zunächst, ob der Parameter \IC{selected} nicht \IC{null} ist, denn sein Parametertyp \IC{bool?} lässt Null-Werte zu.
Durch die Typ-Beförderung ist \IC{selected} innerhalb des Körpers der Fallunterscheidung dann vom Typ \IC{bool} \Z{84-94}. 


Darin wird zunächst der Zustand des \enquote{ViewModels} der \IC{SelectionCard} aktualisiert.
Die \IC{replace}-Methode des \enquote{Builder}-Objektes kann die gesamte Kollektion im \IC{BuiltSet} austauschen, ungeachtet dessen, dass es sich beim Argument selbst nicht um ein \IC{BuiltSet} handelt.
Die \IC{replace}-Methode wandelt das Argument dafür automatisch um.
Durch Zuweisung des neuen Wertes erhält das \enquote{ViewModel} der \IC{SelectionCard} ein neues Ereignis.
Damit werden die \IC{SelectionCard} und der dazugehörige Selektionsbildschirm aktualisiert.
Während der Erstellung dieser Arbeit wurde versucht, die \IC{SelectionCard} als ein \IC{StatefulWidget} zu erstellen.
Mittels \IC{setState} sollte dafür gesorgt werden, dass sowohl \IC{SelectionCard} als auch der Selektionsbildschirm aktualisiert werden.
Doch bei diesem Vorgehen zeichnet sich nur die \IC{SelectionCard} neu.
Der Selektionsbildschirm bleibt unverändert, denn er wird zwar von der \IC{SelectionCard} gebaut, doch ist er nicht tatsächlich Kind der \IC{SelectionCard}.
In Wahrheit ist der Selektionsbildschirm ein Kind von \IC{MaterialApp} -- genau wie \IC{MassnahmenMasterScreen} und \IC{MassnahmenDetailScreen}.

Neben dem \enquote{ViewModel} der \IC{SelectionCard} muss jedoch auch das \enquote{ViewModel} der Eingabemaske aktualisiert werden.
Mit den Rückruffunktionen \IC{onSelect} \Z{90} und \IC{onDeselect} \Z{92} hat die aufrufende Ansicht die Möglichkeit, auf Selektionen zu reagieren.

Schließlich ist noch der \IC{FloatingActionButton} Teil der Unterseite \Z{99-103}.
Mit einem Klick darauf gelangt der Benutzer zurück zur Eingabemaske \Z{100}.


\begin{alexlisting}{Schritt 1}{Die Methode \enquote{createMultipleChoiceSelectionScreen}}
  {Quellcode/Schritt-1/conditional_form/lib/widgets/selection_card.dart}
  {firstline=64, lastline=126}
  \label{lst:Schritt1FunktionCreateMultipleChoiceSelectionScreen}
\end{alexlisting}
 