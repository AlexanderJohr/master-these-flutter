
\ifodd\value{page}\hbox{}\newpage\fi
\section{Serialisierung der Maßnahmenliste}

Damit alle Maßnahmen -- statt nur einer einzigen -- in einer Datei zusammengefasst werden können, müssen die Maßnahmen zunächst zu einer Menge zusammengefasst werden, die ebenfalls serialisierbar ist.
Der Wertetyp \IC{Storage} ist dafür vorgesehen \Lst{\ref{lst:Schritt1WerteTypStorage}}.
Er deklariert allein das \IC{BuiltSet massnahmen} \Z{10}.
Ein \IC{BuiltSet} ist die Abwandlung eines gewöhnlichen \enquote{Sets}, jedoch unter anderem mit der Möglichkeit, es mit einem \enquote{Builder} zu erstellen und das \enquote{Set} zu serialisieren.
% todo high Referenz einfügen
Die Übergabe des Typarguments \IC{<Massnahme>} gewährleistet, dass keine anderen Objekte eingefügt werden können, die weder eine Instanz der Klasse \IC{Massnahme} sind, oder einer Klasse, die von \IC{Massnahmen} erbt.

\begin{alexlisting}{Schritt 1}{Der Wertetyp \enquote{Storage}}
  {Quellcode/Schritt-1/conditional_form/lib/data_model/storage.dart}
  {firstline=9, lastline=10}
  \label{lst:Schritt1WerteTypStorage}
\end{alexlisting}

Der Befehl  \IC{flutter pub run build_runner build} stößt erneut die Quellcodegenerierung für den Wertetyp \IC{Storage} an.

\subsection{Unittest der Serialisierung der Maßnahmenliste}

Nun soll noch überprüft werden, ob die Menge von Maßnahmen mit genau einer eingetragenen Maßnahme korrekt serialisiert.
Auch das wird von einem Unittest überprüft \Lst{\ref{lst:Schritt1MassnahmenSerialisierenOhneFehlerUnitTest}}.
In Zeile 8 wird das leere Objekt \IC{storage} erstellt.
In Zeile 9 wird es dann wiederverwendet, um aufbauend auf der Kopie Änderungen mithilfe der \IC{rebuild}-Methode durchzuführen.

\begin{alexlisting}{Schritt 1}{Unittest der Serialisierung der Maßnahmenliste}
  {Quellcode/Schritt-1/conditional_form/test/data_model/storage_test.dart}
  {firstline=7, lastline=31}
  \label{lst:Schritt1MassnahmenSerialisierenOhneFehlerUnitTest}
\end{alexlisting}

Bei der Instanzvariablen \IC{massnahmen} der Klasse \IC{Storage} handelt es sich um ein \IC{BuiltSet}.
 Der Aufruf von \IC{b.massnahmen}  gibt allerdings nicht dieses \IC{BuiltSet} zurück.
 Wäre es so, so könnte die Operation \IC{add} nicht  darauf angewendet werden.
 Ein \IC{BuiltSet} stellt keine Methoden zur Manipulation des \enquote{Sets} zur Verfügung.
 In Wahrheit gibt der Ausdruck \IC{b.massnahmen} einen \IC{SetBuilder} zurück.
 Das kann im generierten Quellcode nachgesehen werden \LstZ{\ref{lst:Schritt1InstanzvariableMassnahmenGibtEinenSetBuilderZurueck}}{95-96}.

\begin{alexlisting}{Schritt 1}{Instanzvariable \IC{massnahmen} gibt einen \IC{SetBuilder} zurück}
  {Quellcode/Schritt-1/conditional_form/lib/data_model/storage.g.dart}
  {firstline=91, lastline=96}
  \label{lst:Schritt1InstanzvariableMassnahmenGibtEinenSetBuilderZurueck}
\end{alexlisting}

Der \IC{SetBuilder} wiederum erlaubt es, Änderungen am \enquote{Set} vorzunehmen und stellt dafür die -- für ein \enquote{Set} übliche -- Methode \IC{add} bereit.
Im Aufruf von \IC{add} wird dann ein Objekt des Wertetyps \IC{Massnahme} konstruiert \LstZ{\ref{lst:Schritt1MassnahmenSerialisierenOhneFehlerUnitTest}}{10}.
Dazu wird dieses Mal die anonyme Funktion zum Konstruieren der Maßnahme gleich dem Konstruktor übergeben.


Diesmal konstruiert die Methode \IC{serializers.serializeWith}
mit dem generierten Serialisierer \IC{Storage.serializer} ein weiteres \enquote{JSON}-Objekt \Z{12}.\pdfcomment[icon=Note,color=yellow]{29.08.2021 Satz umformuliert}
Genau wie zuvor wird ein \enquote{JSON}-Dokument vorbereitet \Z{14-30},
welches der \enquote{Matcher} \IC{equals} mit dem serialisierten Dokument des soeben konstruierten Objektes \IC{storage} vergleicht \Z{31}.
Das \enquote{JSON}-Dokument unterscheidet sich nur darin,
dass es einen Knoten namens \IC{'massnahmen'} enthält,
der als Wert eine Liste hat.
Die Liste hat nur ein Element.
Weil dieses Mal das Objekt des Typs \IC{Massnahme} nicht direkt zugreifbar ist, muss es zunächst über die Liste der Maßnahmen aus dem \IC{storage}-Objekt abgerufen werden.
Das ist mit dem Befehl \IC{first} möglich,
der das erste Objekt -- und in diesem Fall einzige Objekt -- der Kollektion zurückgibt \Z{17, 21}.
Darüber kann erneut der \IC{guid}  und  das \IC{letztesBearbeitungsDatum} abgerufen werden.

Ein weiterer Unittest überprüft, ob auch die Deserialisierung eines \IC{storage}-Objektes erfolgreich ist.
Er ist in Listing {\ref{lst:Schritt1MassnahmenDeserialisierenOhneFehlerUnitTest} im Anhang \ref{appendix:Schritt1Anhang} auf Seite \pageref{appendix:Schritt1Anhang} zu finden.\pdfcomment[icon=Note,color=yellow]{29.08.2021 Satz umformuliert}
Auch dieser Test ist der Deserialisierung des Objektes des Typs \IC{Massnahme} sehr ähnlich.
Er unterscheidet sich nur darin, dass das \IC{Massnahme}-Objekt in der Liste \IC{massnahmen} des \IC{storage}-Objektes  enthalten ist.

