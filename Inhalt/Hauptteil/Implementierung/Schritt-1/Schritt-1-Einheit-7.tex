\ifodd\value{page}\hbox{}\newpage\fi
\section{Übersichtsbildschirm der Maßnahmen}

Der erste Bildschirm -- für die Übersicht der Maßnahmen -- kann auf das im letzten Schritt erstellte \enquote{Model} zugreifen.
In Listing \ref{lst:Schritt1KlasseMassnahmenMasterScreenStruktur}  ist die Struktur des Übersichtsbildschirms zu sehen.
Über die Route \url{/massnahmen_master} ist der Bildschirm erreichbar \Z{14}. Die \IC{build}-Methode zeichnet die Oberfläche \Z{19-99}.

Mittels \IC{AppState.of(context)} ist nun der Zugriff auf sowohl \enquote{Model} als auch \enquote{ViewModel} möglich. Zur einfacheren Verwendung sind sie als lokale Variablen zwischengespeichert \Z{20-21}.

Das \enquote{Widget} \IC{Scaffold} -- deutsch Gerüst --  stellt  ein grundlegendes Layout mit einer Überschrift und einem Bereich für den Inhalt bereit \Z{23}.
Das \IC{Scaffold} kann auch Mitteilungen an den Benutzer am unteren Bildschirmrand einblenden.

Die Überschrift wird in der sogenannten \IC{AppBar} hinterlegt \Z{24}.
Sie unterstützt weitere Funktionalitäten.
Sollte es sich bei der aktuell besuchten Route  um eine Unterseite handeln, taucht links von der Titel-Überschrift ein Button zum Zurücknavigieren auf.
Weiterhin können rechts von der Titelleiste Aktionsbuttons hinzugefügt werden. Das ist für die Formularanwendung allerdings nicht nötig.

Zusätzlich kann dem \IC{Scaffold} ein Button für die primäre Aktion auf diesem Bildschirm hinzugefügt werden: der sogenannte \IC{FloatingActionButton} \Z{88-97}.
Bei Aktivierung dieses Buttons navigiert die Applikation zur Eingabemaske, um eine neue Maßnahme anzulegen \Z{96}.

Das Eingabeformular sollte den Benutzer auffordern, tatsächlich leere Eingabefelder zu füllen.
Deshalb muss die Aktivierung des Buttons auch das \enquote{ViewModel} neu initialisieren.
Dies geschieht durch Zuweisung einer leeren Maßnahme zur \enquote{Setter}-Methode \IC{vm.model} \Z{95}.
Ohne die Neuinitialisierung würde die Eingabemaske immer die zuletzt eingetragene Maßnahme enthalten, was zu einer großen Verwirrung beim Benutzer führen würde.

Der \IC{FloatingActionButton} erhält den Schlüssel \IC{createNewMassnahmeButtonKey} \Z{89}.
Er ist als \IC{GlobalKey} deklariert \Z{11}.
Er findet beim Integrationstest Anwendung, um den Button zu finden (Siehe \ref{sec:TestDesUebersichtsbildschirms} \enquote{\nameref{sec:TestDesUebersichtsbildschirms}} auf Seite \pageref{sec:TestDesUebersichtsbildschirms}).

Der Inhaltsbereich des \IC{Scaffold} beinhaltet das \enquote{Widget} \IC{StreamBuilder} \Z{27}.
Er kann auf \enquote{Streams} horchen, welche die Ereignisse des Typs \IC{Storage} übermitteln.
Er horcht auf Änderungen im \enquote{Model}, um genau zu sein auf Änderungen des \enquote{Streams} \IC{model.storage} \Z{28}.
Sobald der \IC{StreamBuilder} ein Ereignis erhält, so führt er die Methode aus, die als Argument des Parameters \IC{builder} hinterlegt ist.
Alle \enquote{Widgets} außerhalb davon, wie etwa das \IC{Scaffold}, erhalten dabei keine Aufforderung zum Neuzeichnen, sobald eine Maßnahme hinzugefügt wird.
Das wirkt sich positiv auf die Laufzeitgeschwindigkeit aus.


\ifIncludeFigures
  \begin{listing}[htbp]
    \let\oldtheFancyVerbLine\theFancyVerbLine
    \renewcommand\theFancyVerbLine{%
      \ifnum\value{FancyVerbLine}=31
      \setcounter{FancyVerbLine}{85}
      %\scriptsize\ldots
      \else
      \oldtheFancyVerbLine%
      \fi
    }
    \begin{minted}[firstnumber=11]{dart}
final createNewMassnahmeButtonKey = GlobalKey();

class MassnahmenMasterScreen extends StatelessWidget {
  static const routeName = '/massnahmen_master';

  const MassnahmenMasterScreen({Key? key}) : super(key: key);

  @override
  Widget build(BuildContext context) {
    final model = AppState.of(context).model;
    final vm = AppState.of(context).viewModel;

    return Scaffold(
      appBar: AppBar(
        title: const Text('Maßnahmen Master'),
      ),
      body: StreamBuilder<Storage>(
          stream: model.storage,
          builder: (context, _) {
            return SingleChildScrollView(
              /* ... */
            );
          }),
      floatingActionButton: FloatingActionButton(
          key: createNewMassnahmeButtonKey,
          child: const Icon(
            Icons.post_add_outlined,
            color: Colors.white,
          ),
          onPressed: () {
            vm.model = Massnahme();
            Navigator.of(context).pushNamed(MassnahmenDetailScreen.routeName);
          }),
    );
  }
}
\end{minted}
    \alexlistingcaption{Schritt 1}{Die Struktur der Klasse \enquote{MassnahmenMasterScreen}} {Quellcode/Schritt-1/conditional_form/lib/screens/massnahmen_master.dart}
    \label{lst:Schritt1KlasseMassnahmenMasterScreenStruktur}
  \end{listing}
\fi

\subsection{Auflistung der Maßnahmen im Übersichtsbildschirm}

Der Inhalt der \IC{builder}-Methode ist in Listing \ref{lst:Schritt1DieAusgabeDerMassnahmen} auf Seite \pageref{lst:Schritt1DieAusgabeDerMassnahmen} dargestellt.
Das erste \enquote{Widget} ist ein \IC{SingleChildScrollView} \Z{30}.
\begin{alexlisting}{Schritt 1}{Die Ausgabe der Maßnahmen}
  {Quellcode/Schritt-1/conditional_form/lib/screens/massnahmen_master.dart}
  {firstline=30, lastline=86}
  \label{lst:Schritt1DieAusgabeDerMassnahmen}
\end{alexlisting}

Das Argument \enquote{scrollDirection} ist nicht gefüllt, weshalb die Standardoption -- die vertikale Scrollrichtung -- gewählt wird.
Sollte die Liste der Maßnahmen die Höhe des Fensters überschreiten, so kann der Benutzer vertikal über die Liste scrollen.



Das Kind des Scrollbereichs ist ein \IC{Column}-\enquote{Widget} \Z{31}.
Es zeichnet \enquote{Widgets}, die als Argument des Parameters \IC{children} gesetzt sind, von oben nach unten \Z{33}.
Der Parameter \IC{crossAxisAlignment} gibt an, wie die Kindelemente ausgerichtet sein sollen \Z{32}.
\enquote{crossAxis} bedeutet dabei die entgegengesetzte Achse zur Anzeigerichtung.
Da die \IC{Column} vertikal zeichnet, ist mit \enquote{crossAxis} die horizontale Achse gemeint.
\IC{CrossAxisAlignment.start} beschreibt, dass Elemente entlang der horizontalen Achse an deren Startpunkt auszurichten sind. Dadurch sind alle Elemente der Liste linksbündig.

Zuerst kommt die Auflistung der Maßnahmen, welche als abgeschlossen markiert sind.
Die Überschrift \IC{"Abgeschlossen"} \Z{37} soll einen Abstand von jeweils 16 Pixel in alle Richtungen haben.
Das ermöglicht das \IC{Padding}-\enquote{Widget}  \Z{34-40} mit dem Argument \IC{EdgeInsets.all(16.0)}.
Nach der Überschrift erscheint als zweites Element in der \IC{Column} ein weiterer \IC{SingleChildScrollView} \Z{41-58},
allerdings dieses Mal mit horizontaler Scroll-Richtung \Z{42}.
Sollten die Informationen der Maßnahmen die Breite des Fensters überschreiten, kann der Nutzer von links nach rechts scrollen.

Die Informationen der Maßnahmen werden in einer Tabelle angezeigt.
Dies übernimmt das selbst geschriebene \enquote{Widget} \IC{MassnahmenTable} \Z{45}.
Als erstes Argument erfolgt die Übergabe der anzuzeigenden Maßnahmen aus dem \enquote{Model}.
\IC{storage.value.massnahmen} gibt den aktuellen Wert des \enquote{Streams} des \IC{storage}-Objektes zurück und greift auf die Liste der Maßnahmen zu \Z{46}.
Mit der Methode \IC{where} \Z{47} kann ein Filter auf die Liste angewendet werden.
Die übergebene anonyme Funktion \Z{47-49} überprüft, ob der letzte Status auf \enquote{fertig} gesetzt ist.
Dazu reicht der Vergleich der Abkürzung.
Nur wenn diese Bedingung erfüllt ist,
bleibt die Maßnahme in der gefilterten Kollektion zurück.
Ein solcher Filter gibt ein sogenanntes \enquote{lazy} \enquote{Iterable} zurück.\footcite[Vgl.][]{FlutterWhereMethod}
Erst beim Zugriff auf das Ergebnis findet der Filter Anwendung.
Da es keinen Zwischenspeicher für die gefilterten Elemente gibt,
filtert jeder Zugriff die Elemente neu.
Der Aufruf \IC{toSet} bewirkt allerdings das Speichern der Ergebnisse in einer Menge \Z{50}.
Das \enquote{Widget} \IC{MassnahmenTable} erhält das Resultat,
um es in einer Tabelle anzuzeigen.

Ein weiterer Parameter ist \IC{onSelect} \Z{50}.
Als Argument kann eine Funktion mit genau einem Parameter gesetzt werden.
Sollte der Benutzer in der Tabelle eine Maßnahme auswählen, so löst er damit die Funktion aus.
Der erste Parameter enthält dann die ausgewählte Maßnahme.
Daraufhin soll sich wieder die Eingabemaske öffnen \Z{55-56}.
Dann beinhalten die Eingabefelder jedoch die Werte der ausgewählten Maßnahme.
Um das zu erreichen, reicht eine Zuweisung der Maßnahme an das \enquote{ViewModel} \Z{51}.
Allerdings soll die Maßnahme zuvor ein neues letztes Bearbeitungsdatum mit dem aktuellen Zeitstempel erhalten \Z{51-53}.

Unterhalb der Rubrik der finalen Maßnahmen listet die Übersicht die Maßnahmen, welche sich noch im Entwurf befinden \Z{59-83}.
Daher ist das dritte Element der \IC{Column} wiederum eine Überschrift: \IC{"In Bearbeitung"} \Z{62}, gefolgt von einem weiteren horizontalen Scrollbereich \Z{66-83} mit einer Tabelle von Maßnahmen \Z{70-82}.
Der einzige Unterschied ist hier die Bedingung der Filterfunktion.
Dieses Mal filtert die Kollektion auf Maßnahmen in Bearbeitung \Z{73-74}.

