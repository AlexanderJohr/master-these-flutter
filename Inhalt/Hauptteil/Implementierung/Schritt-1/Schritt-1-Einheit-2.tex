
\ifodd\value{page}\hbox{}\newpage\fi
\subsection{Unittest der Deserialisierung einer Maßnahme}

Analog zur Serialisierung testet der Unittest in Listing \ref{lst:DeserialisierungEinerMassnahmeUnittest} auch die Deserialisierung.
Das \enquote{JSON}-Dokument ist dabei sehr ähnlich und unterscheidet sich lediglich in zwei Details.
Der \IC{'guid'} wird auf einen festen Wert festgelegt \Z{38}.
Im Initialisierungsprozess der Maßnahme wird er dagegen zufällig generiert.
Außerdem wird auch das \IC{letztesBearbeitungsDatum} festgesetzt, nämlich auf die Mikrosekunde \IC{0} \Z{40}.

\begin{alexlisting}{Schritt 1}{Unittest der Deserialisierung einer Maßnahme}
  {Quellcode/Schritt-1/conditional_form/test/data_model/massnahme_test.dart}
  {firstline=36, lastline=56}
  \label{lst:DeserialisierungEinerMassnahmeUnittest}
\end{alexlisting}


Zum Vergleich wird in den Zeilen 46 bis 52 eine Maßnahme über das \enquote{Builder}-Entwurfsmuster generiert
und die gleichen festen Werte werden für die Eigenschaften übergeben.
Dabei ist darauf zu achten, dass die Instanzvariable \IC{letzteBearbeitung} keinen Wert über den Zuweisungs-Operator \IC{=} erhält,
sondern stattdessen die Methode \IC{update} darauf aufgerufen wird \Z{49}.

Da es sich bei der Instanzvariablen \IC{letzteBearbeitung} genauso um ein Objekt eines Wertetyps handelt, ist sie ebenso unveränderlich.
Deshalb kann sie nur über einen \enquote{Builder} manipuliert werden.
Ein Blick in den generierten Quellcode offenbart,
dass es sich bei dem Attribut \IC{letzteBearbeitung} in Zeile 49 nicht um die \enquote{Getter}-Methode des Wertetypen \IC{Massnahme},
sondern in Wahrheit um einen \enquote{Builder} des Typs
\IC{LetzteBearbeitungBuilder} handelt
\LstZ{\ref{lst:Schritt1InstanzvariableLetzteBearbeitungGibtEinenLetzteBearbeitungBuilderZurueck}}{224-225}.

\begin{alexlisting}{Schritt 1}{Instanzvariable \enquote{letzteBearbeitung} gibt einen \enquote{LetzteBearbeitungBuilder} zurück}
  {Quellcode/Schritt-1/conditional_form/lib/data_model/massnahme.g.dart}
  {firstline=216, lastline=227}
  \label{lst:Schritt1InstanzvariableLetzteBearbeitungGibtEinenLetzteBearbeitungBuilderZurueck}
\end{alexlisting}

Die Mikrosekunden für das Datum müssen zunächst in ein Objekt von \IC{DateTime} umgewandelt werden.
Dafür wird der benannte Konstruktor \IC{fromMillisecondsSinceEpoch} von \IC{DateTime} \Z{51} aufgerufen.

\paragraph{Benannte Konstruktoren} In Programmiersprachen wie beispielsweise \enquote{Java} können Methoden überladen werden, indem ihre Methodensignatur geändert wird.
Beim Aufruf der Methode kann über die Anzahl und die Typen der übergebenen Argumente die gewünschte Methode gewählt werden.
Das Gleiche gilt für Konstruktoren.
Wird ein weiterer Konstruktor für eine Klasse in \enquote{Java} benötigt, so besteht einzig und allein die Möglichkeit darin, den Konstruktor zu überladen.
Sowohl überladene Methoden als auch überladene Konstruktoren existieren in \enquote{Dart} nicht.
Wird also in \enquote{Dart} ein alternativer Konstruktor gewünscht, so muss er einen Namen bekommen.
Beim Aufruf des Konstruktors wird dieser Name dann mit einem \IC{.} nach dem Klassennamen angegeben, um den gewünschten Konstruktor zu benennen.\footcite[Vgl.][]{NamedConstructors}


Ganz ähnlich wie bei der Serialisierung wird nun mit \IC{serializers.deserializeWith} unter Angabe des Objektes,
welches die Deserialisierung übernehmen soll -- nämlich wiederum \IC{Massnahme.serializer} -- das \enquote{JSON}-Dokument in ein Objekt des Wertetyps \IC{Massnahme} deserialisiert \Z{53-54}.
Schließlich wird in Zeile 56 das Ergebnis der Deserialisierung mit dem gewünschten Ergebnis verglichen.



Durch Eingabe des Befehls \IC{flutter test test/data_model/massnahme_test.dart} in der Kommandozeile
startet die Ausführung aller Tests in der Testdatei.
Wenn alle Tests erfolgreich ausgeführt wurden und beide Ergebnisse mit den verglichenen Werten übereinstimmen,
erfolgt die Ausgabe: \IC{All tests passed!}.