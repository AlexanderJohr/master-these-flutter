\ifodd\value{page}\hbox{}\newpage\fi

\section{Der Haupteinstiegspunkt}
\label{sec:Haupteinstiegspunkt}

Das Listing \ref{lst:Schritt1DerHaupteinstiegspunkt} zeigt den Haupteinstiegspunkt des Programms.
Darin ist erkennbar, dass sich die Applikation in drei Rubriken einteilen lässt:
\begin{itemize}
  \item das \enquote{Model} \Z{24}
  \item der \enquote{View} \Z{35-38}
  \item das \enquote{ViewModel}. \Z{25}
\end{itemize}



\subsection{Das \enquote{Model-View-ViewModel}-Entwurfsmuster}
\label{sec:ModelViewViewModel}
Das \enquote{Model-View-ViewModel}-Entwurfsmuster -- kurz \enquote{MVVM} -- wurde zunächst von John Gossman für die \enquote{Windows Presentation Foundation} beschrieben.
Das \enquote{Model} beschreibt die Datenzugriffs-Komponente, welche die Daten in relationalen Datenbanken oder hierarchischen Datenstrukturen wie \enquote{XML} oder \enquote{JSON} ablegt.
Der \enquote{View} beschreibt die Oberflächenelemente wie Texteingabefelder und Buttons.
Diese beiden Komponenten sind auch aus dem \enquote{Model-View-Controller}-Entwurfsmuster bekannt.
Das \enquote{Model-View-ViewModel}-Entwurfsmuster ist eine Weiterentwicklung davon und integriert das sogenannte \enquote{ViewModel}.
Es ist dafür zuständig, als Schnittstelle zwischen \enquote{View} und \enquote{Model} zu fungieren.
Die Daten des \enquote{Models} lassen sich in der Regel nicht direkt mit Oberflächenelementen verknüpfen.
Denn es kann notwendig sein, dass die Oberfläche weitere temporäre Daten benötigt, die aber nicht mit den Daten des \enquote{Models} gespeichert werden sollen.
Das \enquote{ViewModel} übernimmt diese Arbeit, indem es die Daten des \enquote{Models} abruft und sie in veränderter Form den Oberflächenelementen zur Verfügung stellt.
Andersherum formt es die Eingaben in der Nutzeroberfläche so um, dass sie im strikten Datenmodell des \enquote{Models} Platz finden.
\footcite[Vgl.][]{IntroductionToModelViewViewModelPatternForBuildingWPFApps}

\IC{MassnahmenModel} \Z{24} verwaltet die eingegebenen Daten der Maßnahmen
und nutzt die Abhängigkeit \IC{MassnahmenJsonFile},
um die Daten auf einem Datenträger als eine \enquote{JSON}-Datei zu speichern.
Somit gehören diese beiden Klassen dem \enquote{Model} an.

\IC{MassnahmenFormViewModel} \Z{25} greift die Daten des \enquote{Models} ab und formt diese um,
sodass sie von dem \enquote{View} \IC{MassnahmenDetailScreen} \Z{38} verändert werden können.
Sollen die Daten gespeichert werden, so stellt \IC{MassnahmenFormViewModel} ebenfalls Methoden zur Verfügung,
um die Daten wieder in das Format des \enquote{Models} einpflegen zu können.

\IC{MassnahmenMasterScreen} \Z{36} stellt eine Ausnahme dar,
denn dieser \enquote{View} präsentiert die Daten aus dem \enquote{Model} ohne eine Schnittstelle über ein \enquote{ViewModel}.
Das ist möglich,
weil die Daten nicht manipuliert,
sondern nur angezeigt werden müssen.

Damit sowohl \enquote{ViewModel} als auch \enquote{Model} von jedem \enquote{View} heraus abrufbar sind,
werden sie in eine Art Service eingefügt \Z{23-25}.
Das \enquote{Widget} \IC{AppState} ist dieser Service.
 Es erhält das \enquote{Model} \Z{24} und das \enquote{ViewModel} \Z{25}  im Konstruktor.
Die Abhängigkeit zum Schreiben des \enquote{Models} in eine \enquote{JSON}-Datei \IC{MassnahmenJsonFile} bekommt das \enquote{Model} ebenfalls im Konstruktor übergeben \Z{24}.
\IC{AppState} ist das erste Element, welches im \enquote{Widget}-Baum auftaucht.
Die gesamte restliche Applikation ist als Kindelement hinterlegt \Z{26}.
Damit können alle \enquote{Widgets} auf den Service zugreifen.

\begin{alexlisting}{Schritt 1}{Der Haupteinstiegspunkt \enquote{MassnahmenFormApp}}
  {Quellcode/Schritt-1/conditional_form/lib/main.dart}
  {firstline=18}
  \label{lst:Schritt1DerHaupteinstiegspunkt}
\end{alexlisting}

\subsection{\enquote{Service Locator} und \enquote{Dependency Inection}}

Das \enquote{service locator pattern} folgt dem Umsetzungsparadigma \enquote{inversion of control} -- deutsch Umkehrung der Steuerung.
Frameworks folgen diesem Muster, indem sie als erweiterbare Skelett-Applikationen fungieren.
Anstatt, dass die Applikation den Programmfluss steuert und dabei selbst Funktionen aufruft, wird die Programmflusssteuerung an das Framework abgegeben und mithilfe von Ereignissen ermöglicht, dass das Framework Funktionen des Nutzers aufruft.
\footcite[Vgl.][]{johnson1988designing}
Im \enquote{service locator pattern} werden Komponenten darüber hinaus zentral registriert und über dieses Register anderen Komponenten zur Interaktion zur Verfügung gestellt.\footcite[Vgl.][]{fowler2004DependencyInjection}
Damit ist es möglich, die Komponenten nicht direkt miteinander verknüpfen zu müssen.
Vor allem für automatisierte Tests ist dies von Vorteil, da solche Abhängigkeiten ausgetauscht werden können, um ganz spezielle Teilfunktionalitäten eines Programms zu testen.
 Mehr dazu im Kapitel \ref{sec:Integrations} \enquote{\nameref{sec:Integrations}} auf Seite \pageref{sec:Integrations}.


Anders als der Name vermuten lässt, steuert \IC{MaterialApp} \Z{26} nicht nur das Aussehen der Applikation im \enquote{Material Design}-Look,
sondern das \enquote{Widget} stellt auch Grundfunktionalitäten einer App wie etwa den \enquote{Navigator} bereit.
Damit hat die Applikation die Möglichkeit -- ähnlich wie bei einer Website -- auf Unterseiten zu navigieren.
Hat der Benutzer die Arbeit in der Unterseite vollendet, so kann der \enquote{Navigator} gebeten werden, zur vorherigen Ansicht zurückzukehren.
 Mit dem Parameter \IC{routes} \Z{34-39} erfolgt die Angabe der Unterseiten, die besucht werden können.
Über \IC{initialRoute} \Z{33} kann die Startseite angegeben werden.


\subsection{Der Service für den applikationsübergreifenden Zustand}
\label{sec:ServiceFuerDenApplikationsuebergreifendenZustand}

Um Daten für alle Kindelemente zugreifbar zu machen, werden die sogenannten \enquote{InheritedWidgets} genutzt.
Der Service \IC{AppState} \Lst{\ref{lst:Schritt1DerServiceAppState}} ist ein solches \enquote{InheritedWidget}.
Im Konstruktor erhält es zunächst den Parameter des Typs \IC{Key} \Z{7}.
Es ist gängige Praxis, in \enquote{Flutter}, jedem \enquote{Widget} im Konstruktor zu ermöglichen, einen solchen Schlüssel zu übergeben.
Es ist jedoch optional.
 Ein solcher Schlüssel kann genutzt werden, um das \enquote{Widget} eindeutig zu identifizieren und es unter anderem über den Schlüssel wiederzufinden.
In den Zeilen 8 und 9 werden das \enquote{Model} und das \enquote{ViewModel} dem Objekt im Konstruktor übergeben.
In den Zeilen 13 und 14 sind sie deklariert.
Der letzte Parameter im Konstruktor ist \IC{child} \Z{10}.
Ihm wird der \enquote{Widget}-Baum übergeben, der Zugriff auf das \enquote{InheritedWidget} haben soll.

Der Aufruf des Basiskonstruktors mit den Argumenten \IC{key} und \IC{child} ist in Zeile 11 zu sehen.
Die Basisklasse von \enquote{InheritedWidget} ist \enquote{ProxyWidget} und erhält exakt dieselben Argumente.
 Das \enquote{ProxyWidget} verwendet das Kindelement, um es im \enquote{Widget}-Baum unterhalb von sich selbst zu zeichnen.
 Eine eigene Methode zum Zeichnen muss also nicht für das \enquote{InheritedWidget} implementiert werden.
Die einzige Methode, welche implementiert werden muss, ist \IC{updateShouldNotify} \Z{23}.
Immer dann, wenn das \enquote{InheritedWidget} selbst aktualisiert wird,
kann es alle \enquote{Widgets}, die davon abhängig sind, benachrichtigen.
 In dem Fall werden diese  \enquote{Widgets} ebenfalls neu gezeichnet.
Für die Formularapplikation ist das allerdings nicht gewünscht.
Die Aktualisierung der Oberfläche soll in den nachfolgenden Schritten selbst kontrolliert werden.
Deshalb erfolgt die Rückgabe \IC{false}, da in Zukunft nicht gewünscht ist, den Applikationszustand komplett auszutauschen.
 Um die  Aktualisierung  der Oberfläche  kümmern sich sowohl \enquote{Model} als auch \enquote{ViewModel}.

Damit ein \enquote{Widget} eine Abhängigkeit von dem \IC{AppState} anmelden kann,
verwendet es in seiner eigenen \enquote{build}-Methode den Ausdruck \IC{dependOnInheritedWidgetOfExactType<AppState>()}.
Der Aufruf der Methode erfolgt auf dem Objekt vom Typ \enquote{BuildContext}.
Weil dieser Kontext bei jedem Zeichnen allen Kindern übergeben wird, kann jedes Kind darüber die Vaterelemente wiederfinden.

Damit der Aufruf leichter lesbar und kürzer ist, empfiehlt das \enquote{Flutter}-Team, eine eigene Klassenmethode zu erstellen, welche die Methode für den Benutzer aufruft \Z{16-17}.
Auch eine Fehlermeldung kann bei dieser Auslagerung geworfen werden, sollte im Kontext kein Objekt des gewünschten Typs vorhanden sein \Z{18}.
Das \enquote{Widget}, welches den \IC{AppState} benötigt, kann dann über die vereinfachte Schreibweise \IC{AppState.of(context)} darauf zugreifen.

\begin{alexlisting}{Schritt 1}{Der Service \enquote{AppState} für den applikationsübergreifenden Zustand}
  {Quellcode/Schritt-1/conditional_form/lib/widgets/app_state.dart}
  {firstline=5}
  \label{lst:Schritt1DerServiceAppState}
\end{alexlisting}

Abbildung \ref{lst:UmlAppState} zeigt die Beziehung zwischen den Bildschirmen und dem \enquote{AppState} auf.
Sowohl \enquote{MassnahmenMasterScreen} und \enquote{MassnahmenDetailScreen} müssen auf \enquote{MassnahmenModel}
und \enquote{MassnahmenFormViewModel} zugreifen können.
Zu diesem Zweck erstellt \enquote{MassnahmenFormApp} den \enquote{AppState}.
Er enthält sowohl \enquote{ViewModel} als auch \enquote{Model}.
Über ihn können beide Bildschirme auf \enquote{Model} und \enquote{ViewModel} zugreifen.





\ifIncludeFigures
  \begin{figure}[h]
    \centering

    \begin{tikzpicture}

      \umlsimpleclass[x=0, y=-2.7]{MassnahmenFormApp}

      \umlsimpleclass[x=6,y=-2.7]{AppState}

      \umlclass[x=7,y=0.22]{MassnahmenModel}{
        jsonFile : MassnahmenJsonFile
      }{
      }
      \umlsimpleclass[x=0,y=-0.5]{MassnahmenFormViewModel}

      \umlsimpleclass[x=0,y=-4.9]{MassnahmenMasterScreen}
      \umlsimpleclass[x=7.0,y=-4.9]{MassnahmenDetailScreen}
      \umluniassoc[]{MassnahmenFormApp}{AppState}

      \umldep[geometry=|-|, anchors=60 and -60]{MassnahmenDetailScreen}{AppState}

      \umldep[geometry=|-|, anchors=90 and -120]{MassnahmenMasterScreen}{AppState}

      \umluniassoc[]{MassnahmenMasterScreen}{MassnahmenDetailScreen}

      \umlaggreg[anchors=60 and 241]{AppState}{MassnahmenModel}
      \umlaggreg[geometry=|-|, anchors=120 and -50]{AppState}{MassnahmenFormViewModel}

    \end{tikzpicture}

    \caption[UML-Diagramm der Beziehungen zwischen den Bildschirmen und dem \enquote{AppState}]{UML-Diagramm der Beziehungen zwischen den Bildschirmen und dem \enquote{AppState}, Quelle: Eigene Abbildung}
    \label{lst:UmlAppState}

  \end{figure}%
\fi
