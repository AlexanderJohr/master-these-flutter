\clearpage 


\subsection{Schritt 3}

In diesem Schritt soll die grundlegende Validierungsfunktion hinzugefügt werden.
Maßnahmen, die als abgeschlossen markiert sind, dürfen keine leeren Eingabefelder enthalten und der Maßnahmentitel darf nicht doppelt belegt sein. \IC{Flutter} stellt das Widget \IC{Form} für die Validierung von Eingabefeldern bereit. 

\subsection{Einfügen des Form-Widgets}

Das Widget \IC{Form} ist ein Container, welcher die Validierung für alle Kinderelemente des Typs \IC{FormField} ausführt. 
Damit es alle Eingabefelder im Formular umgibt, wird es zwischen dem \IC{WillPopScope} und dem \IC{Stack} eingefügt \LstZ{\ref{lst:Schritt3Form}}{168}. 
Darüber hinaus wird die Funktion \IC{saveRecord} durch \IC{saveAndGoBackToOverviewScreenIfValid} ersetzt \Z{167}, welche ebenfalls in diesem Schritt implementiert wird. 
Das \IC{Form}-Widget muss über einen \IC{key} registriert werden \Z{169}, damit auf die Validierungsfunktionen zurückgegriffen werden kann.

\begin{alexlisting}{Schritt 3}{Die Maßnahmencharakteristika Selektionskarten werden ergänzt}
    {Quellcode/Schritt-3/conditional_form/lib/screens/massnahmen_detail/massnahmen_detail.dart}
    {firstline=158, lastline=166, highlightlines={163-165}}
    \label{lst:Schritt3Form}
\end{alexlisting}
 
Die Erstellung des \IC{formKey} findet zu Beginn der \IC{build}-Methode des Eingabeformulars statt \LstZ{\ref{lst:Schritt3FormState}}{20}. 
Der \IC{GlobalKey} identifiziert ein Element, welches durch ein Widget gebaut wurde, über die gesamte Applikation hinweg. 
Es erlaubt darüber hinaus auf das \IC{State}-Objekt zuzugreifen, welches mit dem \IC{StatefulWidget} verknüpft ist. 
Ohne Angabe eines Typparameters kann nur Zugriff auf Funktionen des Typs \IC{State} gewährt werden. 
Doch die gewünschte Methode \IC{validate} ist nur Teil des Typs \IC{FormState}. 
Damit das Element, welches über den \IC{GlobalKey} registriert wurde, auch den \IC{FormState} liefert, kann der entsprechende Typparameter \IC{<FormState>} bei der Erstellung des GlobalKey übergeben werden.

\begin{alexlisting}{Schritt 3}{Die Maßnahmencharakteristika Selektionskarten werden ergänzt}
    {Quellcode/Schritt-3/conditional_form/lib/screens/massnahmen_detail/massnahmen_detail.dart}
    {firstline=17, lastline=20, highlightlines={20}}
    \label{lst:Schritt3FormState}
\end{alexlisting}

\subsection{Validierung des Maßnahmentitels}

Das Eingabefeld für den Maßnahmen-Titel ist ein \IC{TextFormField} \LstZ{\ref{lst:Schritt3createMassnahmenTitelTextFormFieldValidator}}{99}.
Es erbt vom Typ \IC{FormField} und wird daher mit dem Vaterelement \IC{Form} verknüpft. 
Es beinhaltet bereits einen Parameter für die Validierungsfunktion namens \IC{validator} \Z{104}.
Die übergebene Funktion erhält im ersten Parameter den für das Textfeld eingetragenen Wert. 
Die Funktion soll \IC{null} zurückgeben, wenn keine Fehler in der Validierung geschehen sind. 
In jedem anderen Fall soll der Text zurückgegeben werden, der als Fehlermeldung angezeigt werden soll.

\begin{alexlisting}{Schritt 3}{Die Maßnahmencharakteristika Selektionskarten werden ergänzt}
    {Quellcode/Schritt-3/conditional_form/lib/screens/massnahmen_detail/massnahmen_detail.dart}
    {firstline=94, lastline=127, highlightlines={95,100, 104-120}}
    \label{lst:Schritt3createMassnahmenTitelTextFormFieldValidator}
\end{alexlisting}

Sollte der Parameter \IC{null} sein oder aber ein leerer String \Z{105}, so wird die entsprechende Fehlermeldung \IC{'Bitte Text eingeben'} angezeigt \Z{107}.
Damit der Benutzer direkt zu dem fehlerhaften Eingabefeld geführt wird, kann ein Objekt der Klasse \IC{FocusNode} verwendet werden. 
Er wird vor der Konstruktion der Karte erstellt \Z{95} und dem Parameter \IC{focusNode} des \IC{TextFormField} übergeben \Z{100}.
Sollte ein Fehler bei der Validierung gefunden werden, kann mit der Methode \IC{requestFocus} angeordnet werden, den Cursor in das betreffende Feld zu setzen \Z{106}. 
Das sorgt auch dafür, dass das Eingabefeld in den sichtbaren Bereich gerückt wird.

Sollte das Textfeld nicht leer sein, so soll noch überprüft werden, ob der Maßnahmen-Titel bereits vergeben ist. Über das Model kann die Liste der Maßnahmen angefordert werden \Z{110}.
Die Funktion \IC{any} akzeptiert als Argument eine Funktion, die für alle Elemente der Liste ausgeführt wird \Z{110-113}. 
Wenn die Rückgabe der Funktion auch nur in einem Fall \IC{true} ist, so evaluiert auch \IC{any} mit \IC{true}.
Andernfalls ist die Rückgabe \IC{false}. 
Die anonyme Funktion schließt zunächst den Vergleich mit derselben Maßnahme aus, welche sich gerade in Bearbeitung befindet.
Der Vergleich der guid ist dafür ausreichend.
Sollte es eine andere Maßnahme geben, welche den gleichen Titel hat \Z{112-113}, so wird Die lokale Variable \IC{massnahmeTitleDoesAlreadyExists} auf \IC{true} gesetzt.
Der Benutzer bekommt die entsprechende Fehlermeldung \IC{'Dieser Maßnahmentitel ist bereits vergeben'} zu lesen \IC{117}. 
Wenn keine der beiden Fallunterscheidungen das \IC{return}-Statement \Z{107, 117} auslöst, so erfolgt schließlich die Rückgabe von \IC{null}.
In dem Kontext der \IC{validator}-Funktion bedeutet die Rückgabe von \IC{null}, dass die Validierung erfolgreich war.
 

 
Das \IC{Form}-Widget validiert lediglich Kindelemente vom Typ \IC{FormField}.
Dementsprechend wird das Widget \IC{SelectionCard} nicht in die Validierung miteinbezogen.
Es erbt nicht von \IC{FormField}.
Es wäre möglich, eine weitere Klasse zu erstellen, die von \IC{FormField} erbt und alle Parameter für die Erstellung einer Selektions-Karte wiederverwendet.
Doch das würde bedeuten, dass für alle folgenden Schritte jeder weitere Parameter in beiden Konstruktoren der Klassen gepflegt werden müsste.
Um der Arbeit leichter folgen zu können, wurde sich für einen anderen, simpleren Weg entschieden: 
Die Selektionskarte kann ebenso von einem \IC{FormField} umgeben werden \LstZ{\ref{lst:Schritt3buildSelectionCardValidator}}{132-159}, welches die Selektionskarte in der \IC{builder}-Funktion erstellt und an den Parametern nichts ändert, außer einen weiteren hinzuzufügen: der Text für die Fehlermeldung \Z{158}. 
Der erste Parameterder \IC{builder}-Funktion ist das \IC{State}-Objekt das \IC{FormField}.
Es enthält die Getter-Methode \IC{errorText}, die bei gegebenenfalls fehlgeschlagener Validierung die zurückgegebene Fehlermeldung enthält.

\begin{alexlisting}{Schritt 3}{Die Maßnahmencharakteristika Selektionskarten werden ergänzt}
    {Quellcode/Schritt-3/conditional_form/lib/screens/massnahmen_detail/massnahmen_detail.dart}
    {firstline=129, lastline=156, highlightlines={132-144, 154-155}}
    \label{lst:Schritt3buildSelectionCardValidator}
\end{alexlisting}

Die anonyme Funktion, die als Argument dem Parameter \IC{validator} übergeben wird \Z{133-143}, erstellt eine temporäre Menge, die den Wert des \IC{selectionViewModel} enthält, wenn dieser nicht \IC{null} ist, andernfalls ist sie eine leere Menge \Z{134-136}.
Die \IC{validator}-Funktion gibt eine Fehlermeldung zurück, sollte die Menge leer sein \Z{138-140}. 
Ist die Menge dagegen gefüllt, so gibt sie \IC{null} zurück, um mitzuteilen, dass die Validierung erfolgreich war \Z{142}.



Der \IC{errorText} wird im Konstruktor der Klasse \IC{SelectionCard} übergeben \LstZ{\ref{lst:Schritt3errorText}}{29}.
Da er \IC{null} sein darf, ist er mit dem Suffix \IC{?} als Typ mit Null-Zulässigkeit gekennzeichnet \Z{21}. 

\begin{alexlisting}{Schritt 3}{errorText wird der SelectionCard hinzugefügt}
    {Quellcode/Schritt-3/conditional_form/lib/widgets/selection_card.dart}
    {firstline=19, lastline=30, highlightlines={21,29}}
    \label{lst:Schritt3errorText}
\end{alexlisting}

Durch Einfügen einer \IC{Column} zwischen der \IC{Card} \LstZ{\ref{lst:Schritt3ColumnErrorText}}{53} und dem ListTile \Z{57} kann die visuelle Repräsentation der Selektionskarte in der Höhe erweitert werden.
Sollte der \IC{errorText} gesetzt sein \Z{65}, so erscheint unter dem Titel und dem Untertitel eine entsprechende Fehlermeldung \Z{66-71}.
 

\begin{alexlisting}{Schritt 3}{errorText wird ausgegeben}
    {Quellcode/Schritt-3/conditional_form/lib/widgets/selection_card.dart}
    {firstline=53, lastline=74, highlightlines={54-56,64-72}}
    \label{lst:Schritt3ColumnErrorText}
\end{alexlisting}



Oberhalb des vorhandenen \IC{FloatingActionButton} wird nun ein weiterer eingefügt, der zum Speichern des Entwurfs mit der Funktion \IC{saveDraftAndGoBackToOverviewScreen} genutzt werden soll \LstZ{\ref{lst:Schritt3FloatingActionButton}}{210-216}. Der ursprüngliche \IC{FloatingActionButton} versucht die Validierung und die anschließende Speicherung der Maßnahme mithilfe der neuen Funktion \IC{saveAndGoBackToOverviewScreenIfValid} \Z{224}.

\begin{alexlisting}{Schritt 3}{Die Maßnahmencharakteristika Selektionskarten werden ergänzt}
    {Quellcode/Schritt-3/conditional_form/lib/screens/massnahmen_detail/massnahmen_detail.dart}
    {firstline=209, lastline=226, highlightlines={210-219, 224}}
    \label{lst:Schritt3FloatingActionButton}
\end{alexlisting}
 



\begin{alexlisting}{Schritt 3}{Die Maßnahmencharakteristika Selektionskarten werden ergänzt}
    {Quellcode/Schritt-3/conditional_form/lib/screens/massnahmen_detail/massnahmen_detail.dart}
    {firstline=22, lastline=33}
    \label{lst:Schritt3DieMassnahmencharakteristikaSelektionskartenWerdenergaenzt}
\end{alexlisting}


\begin{alexlisting}{Schritt 3}{Die Maßnahmencharakteristika Selektionskarten werden ergänzt}
    {Quellcode/Schritt-3/conditional_form/lib/screens/massnahmen_detail/massnahmen_detail.dart}
    {firstline=84, lastline=92}
    \label{lst:Schritt3DieMassnahmencharakteristikaSelektionskartenWerdenergaenzt}
\end{alexlisting}


\begin{alexlisting}{Schritt 3}{Die Maßnahmencharakteristika Selektionskarten werden ergänzt}
    {Quellcode/Schritt-3/conditional_form/lib/screens/massnahmen_detail/massnahmen_detail.dart}
    {firstline=72, lastline=79}
    \label{lst:Schritt3DieMassnahmencharakteristikaSelektionskartenWerdenergaenzt}
\end{alexlisting}

\begin{alexlisting}{Schritt 3}{Die Maßnahmencharakteristika Selektionskarten werden ergänzt}
    {Quellcode/Schritt-3/conditional_form/lib/screens/massnahmen_detail/massnahmen_detail.dart}
    {firstline=45, lastline=70}
    \label{lst:Schritt3DieMassnahmencharakteristikaSelektionskartenWerdenergaenzt}
\end{alexlisting}




\clearpage 
