\chapter{Schritt 7 Anhang} 
\label{appendix:Schritt7Anhang}


\HP{Ich wurde zwischengeparkt}
Weil es auch möglich sein soll, dass eine Selektions-Karte nicht nur direkt in der Eingabemaske sondern  auch als Kind einer Option anderen Elementes auftaucht, ist ein optionaler Parameter ancestor hinterlegt 49.
\HP{Ich wurde zwischengeparkt}

\HP{Ich wurde zwischengeparkt}
Ist aber die selektions Karte Kind eines anderen Elementes  so soll nur nach Elementen besucht werden, die Kind Elemente des angegebenen Vater Elementes sind. Dies kann mit feiner. Descendant erfolgen. Dazu wird das Vater Element dem Parameter of übergeben. Der Parameter matching erhält wiederum ein Feinde Objekt für das Kindelement. In diesem Fall ist dies erneut find. Text, welcher nach dem Titel der Selektion Skate sucht.
\HP{Ich wurde zwischengeparkt}


\HP{Ich wurde zwischengeparkt}
Um nun das Vater-Element zu finden wird nicht wie zuvor fein. Enzersdorf verwendet.
Während der Erstellung dieser Arbeit wurde versucht lediglich Feinde Objekte zu benutzen. Doch je häufiger ein feinder Objekt in einem anderen verschachtelt wird, desto länger dauert die Suche nach dem gewünschten Element. Ohne Optimierungen dauerte das Finden des Elementes  daher mitunter mehrere Sekunden. Deshalb wurde versucht so häufig es nur geht Alternativen  zugfinder  Objekten zu verwenden. So kann etwa mittels Methode element  und dem Feinde object choice Label das tatsächliche   visuelle Objekt gefunden werden.  mittels Methode findAncestorWidgetOfExactType Kann über die Vater Elemente des Elementes iteriert werden. Sobald das elements mit dem gewünschten typ CheckboxListTile Gefunden wurde  80,  speichert die lokale Variable listtilekey 78  den dem Element hinterlegten key ab.  Der Hintergrund dafür ist, dass Methoden wie expekt, Tab, in schwissel und so weiter nur mit Feinde Objekten funktionieren.   die Methode bikee   konvertiert den Schlüssel in einen fein da, der ein Element über den Schlüssel sucht.  sucht.de ist als Suche nach allen existierenden Textelementen und die anschließende Suche von checkboxlist Teil 1 Enten, die diesen Text enthalten. Mithilfe dieser Optimierung konnten die Tests in ihrer Laufzeit Geschwindigkeit deutlich verbessert werden.

Nachdem überprüft wurde, dass von dem listtile genau ein Element existiert 84, wird es in den sichtbaren Bereich gerückt 88, und anschließend  angeklickt 87, sowie auf abschließen alle Animationen gewartet 88.
\HP{Ich wurde zwischengeparkt}


\HP{Ich wurde zwischengeparkt}
Damit jedoch auch Elemente innerhalb von des List teils gefunden werden können,  wird der feinder, der nach dem key das ist teils sucht zurückgegeben, damit er gegebenenfalls wieder verwendet werden kann.
\HP{Ich wurde zwischengeparkt}
