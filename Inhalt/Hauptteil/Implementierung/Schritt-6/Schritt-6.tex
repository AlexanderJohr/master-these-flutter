\chapter{Schritt 6}
\label{chap:Schritt-6}


\begin{alexfigure}{Inhalt/Hauptteil/Implementierung/Schritt-6/Ü.png}
  {Schritt 6 Eingabemaske}
  {XXX Die Eingabemaske zeigt im Schritt 1 eine Karte zum Selektieren des Status und ein Eingabefeld für den Titel}

  \label{fig:Schritt4Eingabemaske}

\end{alexfigure}

\begin{alexfigure}{Inhalt/Hauptteil/Implementierung/Schritt-6/D.png}
  {Schritt 6 Eingabemaske}
  {XXX Die Eingabemaske zeigt im Schritt 1 eine Karte zum Selektieren des Status und ein Eingabefeld für den Titel}

  \label{fig:Schritt4Eingabemaske}

\end{alexfigure}


\begin{alexfigure}{Inhalt/Hauptteil/Implementierung/Schritt-6/D Z.png}
  {Schritt 6 Eingabemaske}
  {XXX Die Eingabemaske zeigt im Schritt 1 eine Karte zum Selektieren des Status und ein Eingabefeld für den Titel}

  \label{fig:Schritt4Eingabemaske}

\end{alexfigure}

In diesem Schritt soll das Formular um Mehrfachauswahlfelder erweitert werden.
Im Speziellen handelt es sich um das Auswahlfeld Nebenziele.
Es beinhaltet die gleichen Auswahloptionen wie das Auswahlfeld Hauptzielsetzung.

\section{Integrationstest erweitern}

Zunächst wird der Integrationstest um die Auswahl der Nebenziele erweitert \Lst{\ref{lst:Schritt6tabSelectionCard}}.

\begin{alexlisting}{Schritt 6}{XXXX}
  {Quellcode/Schritt-6/conditional_form/integration_test/app_test.dart}
  {firstline=118, lastline=127, highlightlines={121-123}}
  \label{lst:Schritt6tabSelectionCard}
\end{alexlisting}

Zu diesem Zweck löst der Test nach der Auswahl der Hauptzielsetzung \Z{118-119} nun einen Klick auf die Selektionskarte für die Nebenzielsetzung aus \Z{121}.
Dadurch öffnet sich der Auswahlbildschirm,
in welchem die Option \enquote{Bodenschutz} \Z{122} und anschließend die Option \enquote{Klima} \Z{123} gewählt wird.
Mit Auswahl der letzten Option
und durch die damit verbundene Übergabe des Arguments \IC{true} für den optionalen Parameter \IC{tabConfirm} wird der Auswahlbildschirm umgehend wieder geschlossen.
Anschließend erfolgt erneut das Speichern der Maßnahme \Z{125-126}.



Anders als bei den bisherigen Schlüssel-Werte-Paaren innerhalb des Objektes \IC{'massnahmenCharakteristika'} kann der Wert der Nebenziele nicht als einzelner String gespeichert werden \Lst{\ref{lst:Schritt6expectedJson}}.

\begin{alexlisting}{Schritt 6}{XXXX}
  {Quellcode/Schritt-6/conditional_form/integration_test/app_test.dart}
  {firstline=136, lastline=150, highlightlines={140-143}}
  \label{lst:Schritt6expectedJson}
\end{alexlisting}

Bei dem Inhalt der Mehrfachauswahlfelder handelt es sich schließlich um eine Auflistung mehrerer Werte.
Sie wird im erwarteten JSON-Dokument als Array-Literal codiert \Z{140-143}.



\section{Hinzufügen der Menge der Nebenziele}

Für die Menge der Nebenziele müssen keine weiteren Auswahloptionen hinzugefügt werden.
Es werden die gleichen Optionen verwendet,
die auch bei der Menge  mit dem Namen \enquote{Hauptzielsetzung Land} zum Einsatz kommen \LstZ{\ref{lst:Schritt6nebenzielsetzungLandChoices}}{123-124}.

\begin{alexlisting}{Schritt 6}{XXXX}
  {Quellcode/Schritt-6/conditional_form/lib/choices/choices.dart}
  {firstline=219, lastline=224, highlightlines={223-224}}
  \label{lst:Schritt6nebenzielsetzungLandChoices}
\end{alexlisting}

\section{Aktualisierung des Models}

Um die Liste der Nebenziele im Wertetyp \IC{MassnahmenCharakteristika} einzufügen,
kann der Datentyp \IC{BuiltSet} verwendet werden \LstZ{\ref{lst:Schritt6MassnahmenCharakteristika}}{77}.
\begin{alexlisting}{Schritt 6}{XXXX}
  {Quellcode/Schritt-6/conditional_form/lib/data_model/massnahme.dart}
  {firstline=68, lastline=77, highlightlines={77}}
  \label{lst:Schritt6MassnahmenCharakteristika}
\end{alexlisting}
Die Getter-Methode \IC{nebenziele} bedarf keiner Null-Zulässigkeit,
da das Nicht-Vorhandensein von Werten darüber erreicht werden kann,
dass die Menge leer ist.




\section{Aktualisierung der Übersichtstabelle}

Für das Einfügen der Überschrift in der Übersichtstabelle gibt es keine Unterschiede zum bisherigen Vorgehen.
Die Überschrift wird nach der Spaltenüberschrift für die \enquote{Hauptzielsetzung} eingefügt \LstZ{\ref{lst:Schritt6buildColumnHeader}}{28}.

\begin{alexlisting}{Schritt 6}{XXXXX}
  {Quellcode/Schritt-6/conditional_form/lib/widgets/massnahmen_table.dart}
  {firstline=27, lastline=28, highlightlines={28}}
  \label{lst:Schritt6buildColumnHeader}
\end{alexlisting}

Die Anzeige der Werte in den TabellenZellen ist dagegen unterschiedlich \Lst{\ref{lst:Schritt6buildSelectableCell}}.
Dieses Mal handelt es sich um die Aufzählung von mehreren Werten,
weshalb ein \IC{Column}-Widget die einzelnen Einträge untereinander auflistet \Z{46-49}.
Jedes Element des \IC{BuiltSet} \IC{nebenziele} \Z{47} wird über die Methode \IC{map} jeweils in ein Element des Widgets \IC{Text} konvertiert \Z{48}.

\begin{alexlisting}{Schritt 6}{XXXXX}
  {Quellcode/Schritt-6/conditional_form/lib/widgets/massnahmen_table.dart}
  {firstline=42, lastline=50, highlightlines={44-50}}
  \label{lst:Schritt6buildSelectableCell}
\end{alexlisting}

\section{Aktualisierung des ViewModels}

Die Nebenziele werden -- erneut Mit dem Datentyp BuiltSet -- im ViewModel hinzugefügt \Lst{\ref{lst:Schritt6BehaviorSubjectNebenziele}}.
\begin{alexlisting}{Schritt 6}{XXXX}
  {Quellcode/Schritt-6/conditional_form/lib/screens/massnahmen_detail/massnahmen_form_view_model.dart}
  {firstline=18, lastline=21, highlightlines={20-21}}
  \label{lst:Schritt6BehaviorSubjectNebenziele}
\end{alexlisting}

Der benannte Konstruktor \IC{seeded} initialisiert die Instanzvariable mit  einer leeren Menge \Z{20}.
Dafür wird der parameterlose Konstruktor von \IC{BuiltSet} aufgerufen \Z{21}.
Dadurch unterscheidet sich das \IC{BehaviorSubject} von den anderen im ViewModel
und muss dementsprechend bei der Konvertierung zwischen Model in ViewModel gesondert behandelt werden.

Bei Konvertierung von Model in ViewModel sind für alle Auswahloptionen -- genau wie in den Schritten zuvor -- jeweils nur die Abkürzungen verfügbar.
Die Liste der gespeicherten Abkürzungen der Nebenziele muss dementsprechend zuerst in eine Menge von Auswahloptionen konvertiert werden,
bevor sie dem \IC{BuiltSet} übergeben werden kann \Lst{\ref{lst:Schritt6setModel}}.
Die Methode \IC{map} löst das Problem,
indem sie die ihr als Argument übergebene Funktion für jede Abkürzung in der Menge \enquote{Nebenziele} aufruft \Z{65}.
Die übergebene anonyme Funktion konvertiert die Abkürzung in die zugehörige Auswahloption.
Die resultierende Menge kann dem Konstruktor von \IC{BuiltSet} übergeben werden \Z{64-65}.

\begin{alexlisting}{Schritt 6}{XXXX}
  {Quellcode/Schritt-6/conditional_form/lib/screens/massnahmen_detail/massnahmen_form_view_model.dart}
  {firstline=46, lastline=67, highlightlines={64-65}}
  \label{lst:Schritt6setModel}
\end{alexlisting}


Ähnlich verhält es sich bei der Umwandlung des ViewModels in das Model \Lst{\ref{lst:Schritt6nebenzieleSetBuilder}}.

\begin{alexlisting}{Schritt 6}{XXXX}
  {Quellcode/Schritt-6/conditional_form/lib/screens/massnahmen_detail/massnahmen_form_view_model.dart}
  {firstline=69, lastline=81, highlightlines={80-81}}
  \label{lst:Schritt6nebenzieleSetBuilder}
\end{alexlisting}

In diesem Fall muss die Menge der Auswahloptionen der \enquote{Nebenziele} in die entsprechenden Abkürzungen umgewandelt werden,
bevor sie im \enquote{Model} gespeichert wird.
Die Methode \IC{map} er hält zu diesem Zweck erneut eine anonyme Funktion,
welche die Abkürzung der Auswahloptionen abfragt \Z{81}.
Die resultierende Menge wird als Parameter dem Konstruktor \IC{SetBuilder} übergeben \Z{80-81}.
Der \IC{SetBuilder} wiederum kümmert sich um das Bauen des \IC{BuiltSet}, sobald ein Objekt des Typs Massnahme gebaut wird.



\section{Aktualisierung der Eingabemaske}

Unterhalb des Auswahlfeldes für das Hauptziel wird die Selektionkarte für die Nebenziele eingefügt \Lst{\ref{lst:Schritt6buildMultiSelectionCardnebenzielsetzungLandChoices}}.
\begin{alexlisting}{Schritt 6}{XXXX}
  {Quellcode/Schritt-6/conditional_form/lib/screens/massnahmen_detail/massnahmen_detail.dart}
  {firstline=212, lastline=217, highlightlines={215-217}}
  \label{lst:Schritt6buildMultiSelectionCardnebenzielsetzungLandChoices}
\end{alexlisting}

Allerdings handelt es sich dieses Mal um ein Mehrfachauswahlfeld,
weshalb eine neue Methode namens \IC{buildMultiSelectionCard} aufgerufen wird \Z{215-217}.

Da nun zwei Methoden zum Erstellen von Elementen des Widgets \IC{SelectionCard} existieren,
ist es sinnvoll,
den Quellcode zu refaktorisieren,
um redundanten Code zu vermeiden.



Innerhalb der bereits vorhandenen Methode \IC{buildSelectionCard} wird die Routine,
welche für die Validierung das Formulares genutzt wird,
in eine neue Methode namens \IC{validateChoices} \LstZ{\ref{lst:Schritt6buildSelectionCard}}{123-128} ausgelagert.
\begin{alexlisting}{Schritt 6}{XXXX}
  {Quellcode/Schritt-6/conditional_form/lib/screens/massnahmen_detail/massnahmen_detail.dart}
  {firstline=119, lastline=129, highlightlines={123-128}}
  \label{lst:Schritt6buildSelectionCard}
\end{alexlisting}

Sie bekommt die Attribute für den Namen der Menge \Z{124},
die zu validierenden Optionen \Z{125-127}
und schließlich die bisher ausgewählten Optionen aller Auswahlfelder \Z{128} übergeben.
Die ausgelagerte Funktion ist in Anhang \ref{appendix:Schritt6Anhang} in Listing \ref{lst:Schritt6validateChoices} auf Seite \pageref{lst:Schritt6validateChoices} zu finden.


Für die Erstellung der Mehrfachauswahlfelder ist die Methode \IC{buildMultiSelectionCard} zuständig \Lst{\ref{lst:Schritt6buildMultiSelectionCard}}.

\begin{alexlisting}{Schritt 6}{XXXX}
  {Quellcode/Schritt-6/conditional_form/lib/screens/massnahmen_detail/massnahmen_detail.dart}
  {firstline=144, lastline=166}
  \label{lst:Schritt6buildMultiSelectionCard}
\end{alexlisting}

Das übergebene \IC{selectionViewModel} unterstützt mit dem Typometer \IC{BuiltSet} die Auswahl von mehreren Auswahloption \Z{146}.
Bei \IC{selectionViewModel} handelt es sich bereits um eine Menge. Für die Validierung \IC{150} sowie für die Übergabe des initialen Wertes an den Konstruktor der \IC{SelectionCard} \Z{157} ist eine Umwandlung in eine Menge daher nicht mehr nötig.
Dem Konstruktor \IC{SelectionCard} wird weiterhin über den Parameter \IC{multiSelection} mitgeteilt, dass mehr als eine Auswahl zum gewählt werden darf \Z{154}.
Die Methoden \IC{onSelect} und \IC{onDeselect} ersetzen nun nicht mehr den aktuell gespeicherten Wert über eine einfache Zuweisung.
Sie nutzen stattdessen die Methode \IC{rebuild} des \IC{BuiltSet} um ein Element mit Hilfe von \IC{add} hinzuzufügen \Z{160} bzw. mit \IC{remove} Elemente zu entfernen \Z{163}.
Der Methodenaufruf \IC{rebuild} sorgt jedoch nicht für das Hinzufügen oder Löschen am Original-Objekt, sondern erstellt eine Kopie der Liste mit der gewünschten Änderungen.
Deshalb erfolgt eine Zuweisung der Kopie zum Wert des \IC{BehaviorSubject}-Objekts, was wiederum das Auslösen eines neuen Ereignisses bewirkt \Z{158,161}.

\section{Aktualisierung der Selektionskarte}


Diese Selektionskarte wird um die Instanzvariable \IC{multiSelection} erweitert \LstZ{\ref{lst:Schritt6SelectionCard}}{17},
dessen Wert im Konstruktor übergeben wird \Z{27} aber auch ausgelassen werden kann, da der Standardwert \IC{false} angegeben ist.

\begin{alexlisting}{Schritt 6}{XXXX}
  {Quellcode/Schritt-6/conditional_form/lib/widgets/selection_card.dart}
  {firstline=15, lastline=27, highlightlines={17,27}}
  \label{lst:Schritt6SelectionCard}
\end{alexlisting}


Die Rückruffunktion \IC{onChanged} des CheckboxListTile unterscheidet schließlich zwischen Mehrfach- und Einzel-Selektion.
Sollte \IC{multiSelection} mit \IC{true} gesetzt sein \Z{133}, so erstellt die Methode \IC{rebuild} von \IC{BuiltSet} eine Kopie des aktuellen ViewModels der Selektionen.
In der anonymen Funktion,
welche für die Manipulationen an der Kopie genutzt wird,
wird in einer Fallunterscheidung überprüft,
ob das angewählte Element bereits selektiert ist \Z{136}.
Sollte das der Fall sein,
so wird diese bereits selektierte Option, die nun erneut angewählt wurde, mit der Methode \IC{remove} des Builder-Objekts aus dem \IC{BuiltSet} entfernt \Z{137}.
Anderenfalls war die Option nicht selektiert, weshalb sie mit der Methode \IC{add} hinzugefügt wird. 


\begin{alexlisting}{Schritt 6}{XXXX}
  {Quellcode/Schritt-6/conditional_form/lib/widgets/selection_card.dart}
  {firstline=124, lastline=154, highlightlines={133-142, 147}}
  \label{lst:Schritt6XXXX}
\end{alexlisting}