\clearpage
\section{Fazit und Begründung der Auswahl}\pdfcomment[icon=Note,color=blue]{Einschub von hier}
\label{sec:Fazit-und-Begründung-der-Auswahl}



Die beiden Frameworks \enquote{Flutter} und \enquote{React Native} wurden in diesem Kapitel auf die Möglichkeit zur Erstellung von Formularanwendungen geprüft.
Beide Frameworks stellen die nötigen Komponenten dafür bereit.
In \enquote{Flutter} ist der Vorteil zudem, dass die Komponenten in der Standardbibliothek enthalten sind.
In \enquote{React Native} müssen Bibliotheken wie etwa \enquote{Formic}, \enquote{Redux Forms} oder \enquote{React Hook Form} dafür eingebunden werden.

Für einen Vergleich wurde eine minimalistische \enquote{Flutter}-Formularanwendung implementiert,
welche einer Beispielanwendung für die Bibliothek \enquote{React Hook Form} ähnlich ist.
Die \enquote{Flutter}-Anwendung hatte weniger Quellcode,
entscheidender ist aber: Es war ein äußerst geringer Aufwand nötig,
um eine ähnlich benutzerfreundliche Oberfläche zu erstellen.
Auch in diesem Punkt ist \enquote{Flutter} \enquote{React Native} vorzuziehen.

Schließlich wurden beide Frameworks hinsichtlich der Erstellung von automatisierten Tests untersucht.
Die Anleitungen für das Testen in \enquote{React Native} sind auf mehreren Webportalen verteilt und eine Evaluierung der Bibliotheken und Testtreiber durch den Entwickler ist nötig.
Der Aufwand der Einarbeitung in das Testen in \enquote{Flutter} ist dagegen \pdfcomment[icon=Note,color=blue]{Einschub nächster Satz} gering.
Alle Werkzeuge werden vom \enquote{Dart}- und \enquote{Flutter}-Team bereitgestellt.
Die Dokumentation ist umfangreich, folgt jedoch einem roten Faden.
Eine Übersichtsseite fasst die Kerninformationen zusammen und verweist auf die jeweiligen  Seiten für detailliertere Informationen und Übungen.
\pdfcomment[icon=Note,color=blue]{Einschub von hier}

Als Ergebnis dieser drei Vergleiche soll dementsprechend \enquote{Flutter} als Technologie zur Umsetzung der Formularanwendung ausgewählt werden.
\pdfcomment[icon=Note,color=blue]{Einschub bis hier}