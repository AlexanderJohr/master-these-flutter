% Erlaube einfaches copy & paste aus der generierten pdf
\usepackage[T1]{fontenc}
% Erlaube überall underscore zu benutzen ohne \_ schreiben zu müssen
% https://tex.stackexchange.com/questions/20890/define-an-escape-underscore-environment 
\catcode`\_=12
\usepackage[utf8]{inputenc}

% Deutsche Bezeichnungen wie etwa "Inhaltsverzeichnis", statt "Contents"
\usepackage[german]{babel}

% Deutsche Absätze
\parindent 0pt

% Dieses Package ermöglich einfügen von Bildern mittels \includegraphics
\usepackage{graphicx}

% Zwinge Latex zum einfügen von bildern dort wo man sie im tex 
% Dokument hinterlegt mittels \begin{figure}[H]   
\usepackage{float}

% Ändere Highlight Color
\usepackage{color}
\definecolor{insertedCode}{RGB}{190, 230, 190} 

% Syntax Highlighting 
\usepackage{minted}
% Lösche rote Rahmen in minted Quelltext und nutze das Visual Studio Theme
\usemintedstyle{vs}
\setminted{autogobble=true, fontsize=\footnotesize, baselinestretch=1.0, frame=lines, mathescape, linenos, numbersep=5pt, framesep=2mm, breaklines, highlightcolor=insertedCode, breakafter=_@|.}
\renewcommand{\listingscaption}{Code Snippet}
% Veringere Abstand zum Untertitel
\AtEndEnvironment{listing}{\vspace{-14pt}} % <-
% Bei Syntaxfehler keine roten vierecke darum zeichnen
% https://tex.stackexchange.com/questions/343494/minted-red-box-around-greek-characters 


\usepackage{etoolbox}

\makeatletter
\AtBeginEnvironment{minted}{\dontdofcolorbox}
\def\dontdofcolorbox{\renewcommand\fcolorbox[4][]{##4}}
\makeatother



\usepackage[autostyle,german=quotes]{csquotes}

% Für Hintergrundbilder mittels \AddToShipoutPicture
\usepackage{eso-pic}

% Bilder transparent machen
\usepackage{transparent}

% Fallunterscheidungen wie etwa gerade und ungerade Seiten
\usepackage{ifthen}

% Einfügen von Tabellen
\usepackage{tabularx}

% Literaturverzeichnisse verwenden
% backref=true - referenziert von Literaturverzeichnis zurück auf die Seite wo zitiert wurde
\usepackage[style=authortitle, natbib, backend=biber, backref=true]{biblatex}
% Nummer der Zitation nicht hochgestellt, sondern sondern mit
% normaler Textgröße und einem Punkt nach der Nummer
% Beispiel: 1. Teller,”Visibility computations in densely occluded polyhedral environments.“, S. 19
\makeatletter% 
\long\def\@makefntext#1{% 
  \parindent 1em\noindent \hb@xt@ 1.8em{\hss  \hbox {{\normalfont \@thefnmark }}. }#1}%
\makeatother

% Texte für die zurückreferenzierung definieren
\DefineBibliographyStrings{german}{%
  backrefpage = {Zitiert auf der Seite},% originally "cited on page"
  backrefpages = {Zitiert auf den Seiten},% originally "cited on pages"
}

% Links mit Sprungmarken
\usepackage{hyperref}
\hypersetup{
  colorlinks=true,
  linkcolor=black, % Setz die Farbe von Links im 
                   % Inhaltsverzeichnis zurück auf schwarz
  citecolor=black, % Farbe von Literaturlinks = schwarz
  urlcolor=blue    % Farbe von URLs = blau     
}


% Referezieren von Kapiteln mit ihrem Titel
\usepackage{nameref}

% Einfügen von PDFs
\usepackage{pdfpages} 

% Seitenränder
\usepackage{geometry}
\geometry{a4paper, top=20mm, left=40mm, right=20mm, bottom=20mm, headsep=12mm, footskip=12mm}

% Absatzformat
\parskip2ex		% Absatzabsstand	
\parindent0ex		% Absatzeinzug

% Package zum hinzufügen eines Glossars
\usepackage{glossaries}  
\setacronymstyle{long-short}  
\newacronym{eler}{ELER}{Europäischer Landwirtschaftsfond für die Entwicklung des ländlichen Raum}  

% Auf gleiche footnote erneut verweisen mit cleveref
% Achtung: Wenn cleveref nicht zum ende geladen wird gibt es evtl. die Fehlermeldung: Package cleveref Error: cleveref must be loaded after amsmath!.
\usepackage{cleveref}
% Auf erneut verweisen
\crefformat{footnote}{#2\footnotemark[#1]#3} 

% Abbildungsunterschrift anpassen
% https://tex.stackexchange.com/questions/86120/font-size-of-figure-caption-header/86121
\usepackage[font=footnotesize,labelfont=bf]{caption}
% Abstand anpassen
% https://tex.stackexchange.com/questions/45990/how-can-i-modify-vertical-space-between-figure-and-caption
\setlength{\abovecaptionskip}{10pt plus 3pt minus 2pt} 

% Nutze unterschiedliche fußnoten (1,2,3 und a,b,c)
\usepackage[para]{manyfoot}
% \footnoteL{https://google.com}
\DeclareNewFootnote{L}[Roman]
% \footnoteI{https://google.com}
\DeclareNewFootnote{I}[alph]

% fußnoten nacheinander mit Kommas
% \footnote{Einz}\footnote{Zwei}
\usepackage[multiple]{footmisc}



 % https://tex.stackexchange.com/questions/50654/how-to-allow-line-breaks-after-forward-slashes-in-context?rq=1
% enables a break point after /, +, (, ) and -
\usepackage{url}



\newenvironment{alexlisting}[6]
{
    \begin{listing}[h]
        \inputminted[firstline=#4, lastline=#5, highlightlines={#6}] {dart} {#3}

        \caption[#1 #2]
        {#2, Quelle: Eigenes Listing, Datei: \url{#3}}
        }
        {
    \end{listing}
}


\newenvironment{alexlistingzwei}[4]
{
    \begin{listing}[h]
        \inputminted[#4] {dart} {#3}

        \caption[#1 #2]
        {#2, Quelle: Eigenes Listing, Datei: \url{#3}}
        }
        {
    \end{listing}
}