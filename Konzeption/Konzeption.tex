\chapter{Konzeption}
\label{chap:Konzeption}

In diesem Kapitel soll die grafische Benutzeroberfläche konzipiert werden.
Dazu gehören der Übersichtbildschirm,
die Eingabemaske und der Selektionsbildschirm.

\section{Der Übersichtsbildschirm}

Bei Programmstart wird der Benutzer mit dem Übersichtsbildschirm begrüßt \Abb{\ref{fig:KonzeptionÜbersichtsbildschirm}} .
Er listet die bisher eingegebenen Maßnahmen auf. Sie werden in zwei Rubriken gruppiert.
Maßnahmen, in welchen bereits alle Eingabefelder gefüllt und valide sind, werden in der Gruppe \enquote{Abgeschlossen} eingeblendet.
Maßnahmen, welche sich noch im Bearbeitungsmodus befinden,
da ihnen Inhalte in den Eingabefeldern fehlen oder die Eingaben nicht valide sind,
erscheinen in der Rubrik \enquote{in Bearbeitung}. 



\begin{alexfigure}{Konzeption/Uebersichtsbildschirm.pdf}
  {Konzeption des Übersichtsbildschirms}
  {Konzeption des Übersichtsbildschirms}

  \label{fig:KonzeptionÜbersichtsbildschirm}

\end{alexfigure}

Klickt der Benutzer auf den Aktionsbutton unten rechts, so gelangt er auf den zweiten Bildschirm: die Eingabemaske.

\section{Die Eingabemaske}

Die Eingabemaske \Abb{\ref{fig:KonzeptionEingabemaske}} listet die Eingabefelder für die  Eigenschaften der Maßnahme.
Bedeutsam sind hierbei vor allem die Einfach- und Mehrfachauswahlfelder.
Sie werden als Karten dargestellt.
Unterhalb des Titels der Eigenschaft,
dessen Wert mit der Selektionskarte ausgewählt werden soll,
erscheint auch die Anzeige des bisher ausgewählten Wertes
-- für Einfachauswahlfelder --
bzw. der aktuell ausgewählten Werte
-- für Mehrfachauswahlfelder.
Selektionskarten, welche invalide Werte enthalten, werden rot eingefärbt.
Die Selektionskarten können über Überschriften in Gruppen zusammengefasst werden,
um die Übersichtlichkeit zu erhöhen.
Auch Zwischenüberschriften sollen möglich sein.

Die zwei Aktionsbuttons unten rechts ermöglichen das Speichern der Maßnahme.
Der untere der beiden wird dazu verwendet,
die Maßnahme vor dem Speichern zu validieren.
Nur wenn die Validierung erfolgreich ist,
wird die Maßnahme auch gespeichert und der Benutzer gelangt zurück zum Übersichtsbildschirm.
Anderenfalls erhält er eine Fehlermeldung.
Mit dem Button darüber wird die Maßnahme dagegen direkt im Entwurfsmodus abgespeichert,
ohne eine Validierung durchzuführen.
Auch nach Anklicken dieses Buttons gelangt der Nutzer zurück zum Übersichtsbildschirm.
Klickt der Benutzer auf den Zurück-Button oben links im Bildschirm,
so wird versucht,
die Eingabemaske wieder zu verlassen,
sofern dies möglich ist.
Ist die Maßnahme im Entwurfsmodus,
so gelangt der Benutzer mit diesem Button direkt zurück zum Übersichtsbildschirm,
ist die Maßnahme dagegen im Modus \enquote{Abgeschlossen},
so wird zunächst eine Validierung ausgeführt.

\begin{alexfigure}{Konzeption/Eingabemaske.pdf}
  {Konzeption der Eingabemaske}
  {Konzeption der Eingabemaske}

  \label{fig:KonzeptionEingabemaske}

\end{alexfigure}

Mit einem Klick auf die Selektionskarten wird der Benutzer auf den Selektionsbildschirm weitergeleitet, 
um Auswahloptionen für das angeklickte Auswahlfeld auszuwählen.

\section{Der Selektionsbildschirm}

Auf dem Selektionsbildschirm \Abb{\ref{fig:KonzeptionSelektionsbildschirm}} werden alle möglichen Auswahloptionen aufgelistet.
Mit einem Klick darauf kann der Benutzer die Optionen aktivieren und deaktivieren.

Auswahloptionen, welche mit den in anderen Auswahlfeldern ausgewählten Werten nicht kompatibel sind,
erscheinen am Ende der Liste. Sie sind mit einem Kreuz-Symbol gekennzeichnet.
Optionen, welche zuvor angewählt waren und durch eine neue Selektion nun invalide geworden sind,
erscheinen mit einem roten Hintergrund.


Klickt der Benutzer auf den Aktionsbutton unten rechts im Bild oder auf den Zurück-Button oben links im Bild,
so gelangt er zurück zur Eingabemaske.

\begin{alexfigure}{Konzeption/Selektionsbildschirm.pdf}
  {Konzeption des Selektionsbildschirms}
  {Konzeption des Selektionsbildschirms}

  \label{fig:KonzeptionSelektionsbildschirm}

\end{alexfigure}